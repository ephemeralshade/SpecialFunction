\documentclass[b5paper,draft,oneside,openany]{ltjsbook} % 下書き用
%\documentclass[b5paper,oneside,openany]{ltjsbook} % 清書用
\usepackage{../Preamble/preamble}
\begin{document}
\ifdraft{\tableofcontents}{}
この章では,実領域における常微分方程式の解の存在性と一意性に関する命題を扱う。
参考文献は\cite{takano}である。

以下では$t$を実数とし,$x=(x_0,x_1,\dots,x_{n-1})$を$\R^n$の元とする。

まず,記号や用語の準備をする。
$E$を$\R\times\R^{n}$の部分集合とする。
$f\colon E\to\R^n;\pare{t,x}\mapsto \fun{f}{t,x}$を既知関数とする。
以降,
\begin{equation}
    \diff{x}{t}=\fun{f}{t,x}
    \equlabel{ODE_R-微分方程式(ベクトル表記)}
\end{equation}
で表される微分方程式を扱う。
成分毎に書けば,
\begin{equation}
    \diff{x_j}{t}=\fun{f_j}{t,x_0,x_1,\dots,x_{n-1}},\quad\pare{0\leq j\leq n-1}
    \equlabel{ODE_R-微分方程式(成分表記)}
\end{equation}
である。

\begin{defi}\defilabel{ODE_R-微分方程式の解}
    $I$を$\R$上の区間とする。
    $x=\fun{x}{t}$が区間$I$における\equref{ODE_R-微分方程式(ベクトル表記)}の解であるとは,
    \begin{equation}
        \forall j\in\setN,\, \forall t\in I,\, \diff{\fun{x_j}{t}}{t}=\fun{f_j}{t,\fun{x_0}{t},\fun{x_1}{t},\dots,\fun{x_{n-1}}{t}}
    \end{equation}
    が成立することである。
    但し,区間$I$に有限の上界や下界が存在する場合,区間の端での微分は片側微分係数で定める。
    また,$\pare{a,b}\in E$としたとき,条件式$\fun{x}{a}=b$のことを初期条件といい,これを満たす解を,点$\pare{a,b}$を通る解であるという。
\end{defi}

以下では,初期条件も含め,
\begin{equation}
    \diff{x}{t}=\fun{f}{t,x},\ \fun{x}{a}=b
    \equlabel{ODE_R-微分方程式}
\end{equation}
を満たす解に関して考察する。

\section{Picardの逐次近似法}
\begin{thm}[Picardの定理]\thmlabel{ODE_R-Picardの定理}
    $r$, $\rho$を正数とする。
    $\fun{f}{t,x}$が$\R\times\R^n$上の有界閉領域
    \begin{equation}
        E\coloneqq \set{\pare{t,x}\in\R\times\R^n}{\abs{t-a}\leq r,\, \norm{x-b}\leq \rho}
    \end{equation}
    上でLipschitz連続\footnote{\defiref{Lipschitz-Lipschitz連続}参照}であるとする。
    \begin{equation}
        M\coloneqq \max_{\pare{t,x}\in E}\bra{\norm{\fun{f}{t,x}}},\
        r'\coloneqq \min\bra{r,\frac{\rho}{M}}
    \end{equation}
    としたとき,\equref{ODE_R-微分方程式}を満たす解が区間$I'\coloneqq [a-r',a+r']$において一意的に存在する。
    \begin{proof}
        \equref{ODE_R-微分方程式}を満たすことと,
        \begin{equation}
            \fun{x}{t}=b+\int_a^t ds\, \fun{f}{s,\fun{x}{s}}
            \equlabel{ODE_R-積分方程式}
        \end{equation}
        は同値である。
        実際,\equref{ODE_R-積分方程式}を満たす$x=\fun{x}{t}$の第$j$成分は,微積分学の基本定理により,
        \begin{equation}
            \diff{\fun{x_j}{t}}{t}=\diff{}{t}\pare{b_j+\int_a^t ds\, \fun{f_j}{s,\fun{x}{s}}}
            =\fun{f_j}{t,\fun{x}{t}}
        \end{equation}
        を満たし,
        \begin{equation}
            \fun{x_j}{a}=b_j+\int_a^a ds\, \fun{f_j}{s,\fun{x}{s}}=b_j
        \end{equation}
        も満たす。
        ベクトルでまとめて表せば\equref{ODE_R-微分方程式}を満たすことに他ならない。
    \end{proof}
\end{thm}


\end{document}
