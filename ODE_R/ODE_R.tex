\documentclass[b5paper,draft,oneside,openany]{ltjsbook} % 下書き用
%\documentclass[b5paper,oneside,openany]{ltjsbook} % 清書用
\usepackage{../Preamble/preamble}
\begin{document}
\ifdraft{\tableofcontents}{}
この章では,実領域における常微分方程式の解の存在性と一意性に関する命題を扱う。
参考文献は\cite{takano}である。

まず,記号や用語の準備をする。
以下では$t$を実数とし,$x=(x^0,x^1,\dots,x^{n-1})$を$\R^n$の元とする。
$E$を$\R\times\R^{n}$の部分集合とし,
\begin{equation}
    f\colon E\to\R^n;\pare{t,x}\mapsto \fun{f}{t,x}=(\fun{f^0}{t,x},\fun{f^1}{t,x},\dots,\fun{f^{n-1}}{t,x})
\end{equation}
を既知関数とする。
以降,
\begin{equation}
    \diff{x}{t}=\fun{f}{t,x}
    \equlabel{ODE_R-微分方程式(ベクトル表記)}
\end{equation}
で表される微分方程式を扱う。
成分毎に書けば,
\begin{equation}
    \diff{x^j}{t}=\fun{f^j}{t,x^0,x^1,\dots,x^{n-1}},\quad\pare{0\leq j\leq n-1}
    \equlabel{ODE_R-微分方程式(成分表記)}
\end{equation}
である。

\begin{defi}\defilabel{ODE_R-微分方程式の解}
    $I$を$\R$上の区間とする。
    $x=\fun{x}{t}$が区間$I$における\equref{ODE_R-微分方程式(ベクトル表記)}の解であるとは,
    \begin{equation}
        \forall j\in\setN,\, \forall t\in I,\, \diff{\fun{x^j}{t}}{t}=\fun{f^j}{t,\fun{x^0}{t},\fun{x^1}{t},\dots,\fun{x^{n-1}}{t}}
    \end{equation}
    が成立することである。
    但し,区間$I$に有限の上界や下界が存在する場合,区間の端での微分は片側微分係数で定める。
    また,$\pare{a,b}\in E$としたとき,条件式$\fun{x}{a}=b$のことを初期条件といい,これを満たす解を,点$\pare{a,b}$を通る解であるという。
\end{defi}

以下では,初期条件も含め,
\begin{equation}
    \diff{x}{t}=\fun{f}{t,x},\ \fun{x}{a}=b
    \equlabel{ODE_R-微分方程式}
\end{equation}
を満たす解に関して考察する。

\section{Picardの逐次近似法}
\begin{thm}[Gr\"{o}nwall-Bellmanの不等式]\thmlabel{ODE_R-Gronwall-Bellmanの不等式}
    $a<a'$とし,$I\coloneqq [a,a']$とする。
    関数$X\colon I\to \R$は$I$上で連続とする,
    また,関数$c,L\colon I\to \R$は$I$上連続かつ非負であるとする。
    \begin{equation}
        \forall t\in I,\ 0\leq \fun{X}{t}\leq \fun{c}{t} + \int_a^t ds\, \fun{L}{s}\fun{X}{s}
        \equlabel{ODE_R-Gronwall-Bellmanの不等式-仮定}
    \end{equation}
    が成り立つとき,
    \begin{equation}
        \forall t\in I,\ 0\leq \fun{X}{t}\leq \fun{c}{t}+\int_a^t ds\, \fun{c}{s}\fun{L}{s}\fun{\exp}{\int_s^t du\, \fun{L}{u}}
        \equlabel{ODE_R-Gronwall-Bellmanの不等式}
    \end{equation}
    が成立する。
    \begin{proof}
        $y\colon I\to \R$を,
        \begin{equation}
            \fun{y}{t}\coloneqq \int_a^t ds\,\fun{L}{s}\fun{X}{s}
        \end{equation}
        と定めると,\equref{ODE_R-Gronwall-Bellmanの不等式-仮定}から
        \begin{equation}
            0\leq \fun{X}{t}\leq \fun{c}{t}+\fun{y}{t}
            \equlabel{ODE_R-Gronwall-Bellmanの不等式-yの評価式}
        \end{equation}
        となる。
        微積分学の基本定理により,
        \begin{equation}
            \diff{\fun{y}{t}}{t}=\fun{L}{t}\fun{X}{t}
        \end{equation}
        である。
        $\fun{L}{t}$は$I$上で非負であるから,\equref{ODE_R-Gronwall-Bellmanの不等式-仮定}に$\fun{L}{t}$を掛けることで
        \begin{equation}
            \forall t\in I,\ \diff{\fun{y}{t}}{t} - \fun{L}{t}\fun{y}{t}\leq \fun{L}{t}\fun{c}{t}
        \end{equation}
        が得られる。
        ここで,両辺に$\exp\spare{-\int_a^t du\, \fun{L}{u}}(> 0)$を掛けると,$I$上の任意の$t$に対し,
        \begin{equation}
            \fun{\exp}{-\int_a^t du\, \fun{L}{u}}\diff{\fun{y}{t}}{t}
            -\fun{L}{t}\fun{\exp}{-\int_a^t du\, \fun{L}{u}}\fun{y}{t}
            \leq \fun{L}{t}\fun{c}{t}\fun{\exp}{-\int_a^t du\, \fun{L}{u}}
        \end{equation}
        が得られる。
        この式の左辺は,
        \begin{equation}
            \diff{}{t}\squ{\fun{\exp}{-\int_a^t du\, \fun{L}{u}} \fun{y}{t}}
            =\fun{\exp}{-\int_a^t du\, \fun{L}{u}}\diff{\fun{y}{t}}{t}
            -\fun{L}{t}\fun{\exp}{-\int_a^t du\, \fun{L}{u}}\fun{y}{t}
        \end{equation}
        と書けるから,
        \begin{equation}
            \forall t\in I,\ \diff{}{t}\squ{\fun{\exp}{-\int_a^t du\, \fun{L}{u}} \fun{y}{t}}
            \leq \fun{L}{t}\fun{c}{t}\fun{\exp}{-\int_a^t du\, \fun{L}{u}}
        \end{equation}
        が成立する。
        この式の両辺を$a$から$t$まで定積分することを考える。
        左辺について,定義から$\fun{y}{a}=0$であることに注意すると,
        \begin{align}
            &\int_a^t ds\, \diff{}{s}\squ{\fun{\exp}{-\int_a^s du\, \fun{L}{u}} \fun{y}{s}}
            \\
            &=\squ{\fun{\exp}{-\int_a^s du\, \fun{L}{u}} \fun{y}{s}}_a^t
            =\fun{\exp}{-\int_a^t du\, \fun{L}{u}} \fun{y}{t}
        \end{align}
        となる。
        右辺は,
        \begin{equation}
            \int_a^t ds\, \fun{L}{s}\fun{c}{s}\fun{\exp}{-\int_a^s du\, \fun{L}{u}}
        \end{equation}
        であるから,
        \begin{equation}
            \fun{\exp}{-\int_a^t du\, \fun{L}{u}} \fun{y}{t}\leq 
            \int_a^t ds\, \fun{L}{s}\fun{c}{s}\fun{\exp}{-\int_a^s du\, \fun{L}{u}}
        \end{equation}
        が得られる。
        この式の両辺に$\fun{\exp}{\int_a^t du\, \fun{L}{u}}(>0)$を掛ける。
        左辺は$\fun{y}{t}$になり,右辺は,
        \begin{align}
            &\fun{\exp}{\int_a^t du\, \fun{L}{u}} \int_a^t ds\, \fun{L}{s}\fun{c}{s}\fun{\exp}{-\int_a^s du\, \fun{L}{u}}
            \\
            &=\int_a^t ds\, \fun{L}{s}\fun{c}{s}\fun{\exp}{\int_s^t du\, \fun{L}{u}}
        \end{align}
        となる。
        ここで,指数法則と
        \begin{equation}
            \int_a^t du\, \fun{L}{u}-\int_a^s du\, \fun{L}{u}
            =\int_a^t du\, \fun{L}{u}+\int_s^a du\, \fun{L}{u}
            =\int_s^t du\, \fun{L}{u}
        \end{equation}
        を用いた。

        これらにより,
        \begin{equation}
            \fun{y}{t}\leq \int_a^t ds\, \fun{L}{s}\fun{c}{s}\fun{\exp}{\int_s^t du\, \fun{L}{u}}
        \end{equation}
        が導かれ,\equref{ODE_R-Gronwall-Bellmanの不等式-yの評価式}と併せると,
        \begin{equation}
            0\leq \fun{X}{t}\leq \fun{c}{t}+\fun{y}{t}\leq \fun{c}{t}+\int_a^t ds\, \fun{c}{s}\fun{L}{s}\fun{\exp}{\int_s^t du\, \fun{L}{u}}
        \end{equation}
        が従い,これは\equref{ODE_R-Gronwall-Bellmanの不等式}の成立を意味する。
    \end{proof}
\end{thm}

\begin{cor}\corlabel{ODE_R-Gronwall-Bellmanの不等式の系}
    \thmref{ODE_R-Gronwall-Bellmanの不等式}において,$c$, $L$を単なる非負定数とすると,
    \begin{equation}
        \forall t\in I,\ 0\leq \fun{X}{t}\leq c+L\int_a^t ds\, \fun{X}{s}
    \end{equation}
    が成立するとき,
    \begin{equation}
        \forall t\in I,\ 0\leq \fun{X}{t}\leq ce^{\pare{t-a}L}
    \end{equation}
    が成り立つ。
    \begin{proof}
        \equref{ODE_R-Gronwall-Bellmanの不等式}の右辺の積分は,
        \begin{align}
            &\int_a^t ds\, cL\fun{\exp}{\int_s^t du\, L}
            =cL\int_a^t ds\, e^{\pare{s-t}L}
            =cLe^{-tL}\cdot\squ{-\frac{e^{-sL}}{L}}_a^t
            \\
            &=-ce^{tL}\pare{e^{-tL}-e^{-aL}}
            =-c + ce^{\pare{t-a}L}
        \end{align}
        となるから,\equref{ODE_R-Gronwall-Bellmanの不等式}より,
        \begin{equation}
            \forall t\in I,\
            0\leq \fun{X}{t}\leq c + \pare{-c + ce^{\pare{t-a}L}} = ce^{\pare{t-a}L}
        \end{equation}
        が従う。
    \end{proof}
\end{cor}

\thmref{ODE_R-Gronwall-Bellmanの不等式}から得られた\corref{ODE_R-Gronwall-Bellmanの不等式の系}は,後述する\thmref{ODE_R-Picardの定理}の証明において,Picardの逐次近似法によって構成した解が唯一の解であることを示す際に有用である。


\begin{thm}[Picardの定理]\thmlabel{ODE_R-Picardの定理}
    $r$, $\rho$を正数とする。
    $\fun{f}{t,x}$が$\R\times\R^n$上の有界閉領域
    \begin{equation}
        E\coloneqq \set{\pare{t,x}\in\R\times\R^n}{\abs{t-a}\leq r,\, \norm{x-b}\leq \rho}
        \equlabel{ODE_R-Eの定義}
    \end{equation}
    上で$x$に関してLipschitz連続\footnote{\defiref{Lipschitz-Lipschitz連続}参照},すなわち,
    \begin{equation}
        \exists L\in\Rnonnega;\, \forall t\in\R,\,\forall x,\forall y\in\R^n,\, \squ{\pare{t,x},\pare{t,y}\in E\ \Rightarrow\ \pare{\norm{\fun{f}{t,x}-\fun{f}{t,y}}\leq L\norm{x-y}}}
        \equlabel{ODE_R-Lipschitz条件}
    \end{equation}
    であるとする\footnote{$\fun{f}{t,x}$と$\fun{f}{t,y}$で$t$は共通のものであることに注意。}。
    \begin{equation}
        M\coloneqq \max_{\pare{t,x}\in E}\bra{\norm{\fun{f}{t,x}}},\
        r'\coloneqq \min\bra{r,\frac{\rho}{M}}
        \equlabel{ODE_R-Mrの定義}
    \end{equation}
    としたとき,\equref{ODE_R-微分方程式}を満たす解が区間$I'\coloneqq [a-r',a+r']$において一意的に存在する。
    \begin{policy}
        \equref{ODE_R-微分方程式}を満たすことと,
        \begin{equation}
            \fun{x}{t}=b+\int_a^t ds\, \fun{f}{s,\fun{x}{s}}
            \equlabel{ODE_R-積分方程式}
        \end{equation}
        は同値である。
        実際,\equref{ODE_R-積分方程式}を満たす$x=\fun{x}{t}$の第$j$成分は,微積分学の基本定理により,
        \begin{equation}
            \diff{\fun{x^j}{t}}{t}=\diff{}{t}\pare{b^j+\int_a^t ds\, \fun{f^j}{s,\fun{x}{s}}}
            =\fun{f^j}{t,\fun{x}{t}}
        \end{equation}
        を満たし,
        \begin{equation}
            \fun{x^j}{a}=b^j+\int_a^a ds\, \fun{f^j}{s,\fun{x}{s}}=b^j
        \end{equation}
        も満たす。
        ベクトルでまとめて表せば\equref{ODE_R-微分方程式}を満たすことに他ならない。
        
        \equref{ODE_R-積分方程式}の解の存在性を示すために,Picardの逐次近似法によって区間$I'$上で\equref{ODE_R-積分方程式}の解を構成する。
        Picardの逐次近似法では,
        \begin{equation}
            \forall t\in I',\ \norm{\fun{x_0}{t}-b}\leq \rho
        \end{equation}
        を満たすような連続なベクトル値関数$\fun{x_0}{t}$を用いて,
        \begin{equation}
            \fun{x_{m+1}}{t}\coloneqq b + \int_a^t ds\, \fun{f}{s,\fun{x_{m}}{s}}
            \equlabel{ODE_R-Picardの逐次近似法定義}
        \end{equation}
        によって$\fun{x_1}{t}$, $\fun{x_2}{t},\dots$を定義する。
        $\fun{x_{m+1}}{t}$の右辺の積分の中に$\fun{x_m}{t}$があるため,$\fun{x_m}{t}$が$f$の定義域に収まっていることを確認する必要がある。
        
        そして,このように構成した関数列$\seq{x_m}{m\in\N}$の$m\to\infty$における各点極限関数$\fun{x}{t}$が\equref{ODE_R-微分方程式}の解になっていることを示す。
        そのためには,\equref{ODE_R-Lipschitz条件}のLipschitz連続の条件と,\thmref{一様収束-WeierstrassのM判定法}を用いて$\fun{x_m}{t}$が各点極限関数$\fun{x}{t}$に一様収束することを確かめる。
        一様収束することが確かめられたら,\equref{ODE_R-Picardの逐次近似法定義}の両辺で$m\to\infty$の極限を取る。
        左辺は$\fun{x}{t}$になる。
        右辺は,$f$の連続性と\propref{一様収束-積分と極限の順序交換}を用いて積分と極限の順序交換を行って
        \begin{equation}
            \lim_{m\to\infty}\squ{b + \int_a^t ds\, \fun{f}{s,\fun{x_{m}}{s}}}
            =b + \int_a^t ds\, \lim_{m\to\infty}\fun{f}{s,\fun{x_{m}}{s}}
            =b + \int_a^t ds\, \fun{f}{s,\fun{x}{s}}
        \end{equation}
        とすれば,Picardの逐次近似法\equref{ODE_R-Picardの逐次近似法定義}で定義された関数列が\equref{ODE_R-積分方程式}の解であることが分かる。

        そして,構成した解が唯一の解であることを\thmref{ODE_R-Gronwall-Bellmanの不等式}から得られた\corref{ODE_R-Gronwall-Bellmanの不等式の系}の不等式を用いて示す。
    \end{policy}
    \begin{proof}
        方針で述べたように,\equref{ODE_R-微分方程式}の微分方程式と等価な,\equref{ODE_R-積分方程式}の積分方程式を考察する。

        まずは解の存在性を示すためのPicardの逐次近似法について述べる。
        $\fun{x_0}{t}$は,
        \begin{equation}
            \forall t\in I',\ \norm{\fun{x_0}{t}-b}\leq \rho
            \equlabel{ODE_R-x0の不等式}
        \end{equation}
        を満たす任意の連続関数とする。
        このとき,
        \begin{equation}
            \fun{x_1}{t}\coloneqq b + \int_a^t ds\, \fun{f}{s,\fun{x_0}{s}}
        \end{equation}
        によって$\fun{x_1}{t}$を定めると,
        \begin{equation}
            \forall t\in I',\ \norm{\fun{x_1}{t}-b}\leq \rho
        \end{equation}
        も満たすことを示す。
        実際,
        \begin{equation}
            \norm{\fun{x_1}{t}-b}=\norm{\int_a^t ds\, \fun{f}{s,\fun{x_0}{s}}}
            \leq \abs{\int_a^t ds\, \norm{\fun{f}{s,\fun{x_0}{s}}}}
            \leq M\abs{t-a}\leq Mr'\leq \rho
            \equlabel{ODE_R-x1の不等式}
        \end{equation}
        である。
        
        \equref{ODE_R-x1の不等式}のそれぞれの不等号の理由を順に述べる。
        最初の不等号では,\propref{実ノルム-積分の不等式}の不等式を用いており,今回は$t\in I'=[a-r',a+r']$であるから,$t<a$の場合もあるので最初の不等号の式には絶対値が付く。
        二番目の不等号では\equref{ODE_R-Mrの定義}における$M$の定義から,
        \begin{equation}
            \forall \pare{t,x}\in E,\
            \norm{\fun{f}{t,x}}\leq M
        \end{equation}
        が成立することと\equref{実ノルム-積分絶対値不等式}の不等式を用いている。
        最後の不等号では,\equref{ODE_R-Mrの定義}における$r'$の定義から,$r'\leq \rho/M$であり,これを$M$で払うことで$Mr'\leq \rho$である。

        これにより,$t\in I'$のとき$\norm{\fun{x_1}{t}-b}\leq \rho$である。
        次に,\equref{ODE_R-x0の不等式}を満たす$\fun{x_0}{t}$を用いて,
        \begin{equation}
            \fun{x_{m+1}}{t} \coloneqq b + \int_a^t ds\, \fun{f}{s,\fun{x_m}{t}},\ \pare{m\in\N,\, t\in I'}
            \equlabel{ODE_R-xmの定義式}
        \end{equation}
        によって$\fun{x_1}{t}$, $\fun{x_2}{t}$, $\fun{x_3}{t},\dots$を定める。
        $f$の定義域は\equref{ODE_R-Eの定義}で定義された$E$上であるから,右辺の積分が定義されるためには,
        \begin{equation}
            \forall m\in \N,\ \forall t\in I',\ \norm{\fun{x_m}{t}-b}\leq \rho 
            \equlabel{ODE_R-xmの不等式}
        \end{equation}
        である必要があるが,これは次に述べるように満たされている。
        適当な自然数$m$で,
        \begin{equation}
            \forall t\in I',\ \norm{\fun{x_m}{t}-b}\leq \rho 
        \end{equation}
        が満たされていると仮定すると,\equref{ODE_R-x1の不等式}の議論と全く同じように任意の$t\in I'$に対して
        \begin{equation}
            \norm{\fun{x_{m+1}}{t}-b}=\norm{\int_a^t ds\, \fun{f}{s,\fun{x_m}{s}}}
            \leq \abs{\int_a^t ds\, \norm{\fun{f}{s,\fun{x_m}{s}}}}
            \leq M\abs{t-a}\leq Mr'\leq \rho
        \end{equation}
        であり,$\fun{x_1}{t}$についての成立は\equref{ODE_R-x1の不等式}で確かめているので,数学的帰納法によって\equref{ODE_R-xmの不等式}は成立する。

        以上の議論により,\equref{ODE_R-xmの定義式}によって任意の自然数$m$に対して$\fun{x_m}{t}$が定まる。
        次に,\equref{ODE_R-xmの定義式}で定義された$\fun{x_m}{t}$は,$m\to\infty$の極限で\equref{ODE_R-積分方程式}の解になることを示す。
        そのためには,$\fun{x_m}{t}$が一様収束することを示し,$f$の連続性から極限と積分を入れ替えられることを用いる。
    \end{proof}
\end{thm}


\end{document}
