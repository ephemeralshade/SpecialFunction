% \documentclass[b5paper,draft,oneside,openany]{ltjsbook} % 下書き用
\documentclass[b5paper,oneside,openany]{ltjsbook} % 清書用
\usepackage{../Preamble/preamble}
\begin{document}
\ifdraft{\tableofcontents}{}
\section{ノルム}
ノルムに関する以下の定義や命題は\cite{nomura}に拠る。

\begin{defi}[ノルム]
    $V$を$\CC$上のベクトル空間とする。
    関数$\norm{\dummy}\colon V\to \R$が以下の条件を満たすとき,$\norm{x}$を$x$のノルムという:
    \begin{enumerate}[label=(\roman*)]
        \item (正値性) $\forall x\in V,\ \norm{x}\geq 0$
        \item (一意性) $\forall x\in V,\ \bsqu{\norm{x}=0\ \Longleftrightarrow\ x=0}$
        \item (同次性) $\forall k\in \CC,\ \forall x\in V,\ \norm{kx}=\abs{k}\norm{x}$
        \item (三角不等式) $\forall x, \forall y\in V,\ \norm{x+y}\leq \norm{x}+\norm{y}$
    \end{enumerate}
    ノルムの定義されたベクトル空間をノルム空間という。
\end{defi}


\begin{prop}[三角不等式]\proplabel{ノルム三角不等式}
    ノルムは次の性質も満たす:
    \begin{equation}
        \forall x, \forall y\in V,\ \abs{\norm{x}-\norm{y}}\leq \norm{x-y}.
    \end{equation}
    \begin{proof}
        まず,$\norm{x}=\norm{\pare{x-y}+y}\leq \norm{x-y}+\norm{y}$
        から$\norm{x}-\norm{y}\leq \norm{x-y}$
        が従う。
        同様に
        \begin{equation}
            \norm{y}=\norm{\pare{y-x}+x}\leq \norm{\pare{-1}\pare{x-y}}+\norm{x}=\abs{-1}\norm{x-y}+\norm{x}=\norm{x-y}+\norm{x}
        \end{equation}
        なので,$\norm{y}-\norm{x}\leq \norm{x-y}$である。
        また,絶対値の特徴付け$\abs{a}= \max\bra{a,-a}$
        を用いると,$\abs{\norm{x}-\norm{y}}= \max\bra{\norm{x}-\norm{y},\norm{y}-\norm{x}}\leq \norm{x-y}$である。
        これらにより,$\abs{\norm{x}-\norm{y}}\leq \norm{x-y}$が成立する。
    \end{proof}
\end{prop}

\begin{cor}
    \propref{ノルム三角不等式}において,$y\mapsto -y$として,元の三角不等式と併せれば,
    \begin{equation}
        \abs{\norm{x}-\norm{y}}\leq \norm{x+y}\leq \norm{x}+\norm{y}
    \end{equation}
    が従う。
\end{cor}

\begin{cor}[ノルムの連続性]\corlabel{ノルム連続性}
    ノルムは連続関数である。
    \begin{proof}
        示すべきことは以下の通り:
        \begin{equation}
            \forall x,\forall y\in V,\ \forall \varepsilon\in\R,\, \squ{
                \varepsilon>0 \, \Longrightarrow\, \pare{
                    \exists \delta\in \R;\ \bsqu{
                        \delta>0\, \land\, \pare{
                            \norm{x-y}<\delta\, \Longrightarrow\, \abs{\norm{x}-\norm{y}}<\varepsilon
                        }
                    }
                }
            }.
        \end{equation}

        実際,$\delta\coloneqq \varepsilon$と取れば,\propref{ノルム三角不等式}により,$\abs{\norm{x}-\norm{y}}\leq \norm{x-y}<\delta =\varepsilon$であるから,ノルムは連続関数である。
    \end{proof}
\end{cor}

\begin{defi}[ノルムの同値性]
    以下が成立するとき,$V$上の2つのノルム$\norm{\dummy}$, $\norm{\dummy}_1$が同値であるという:
    \begin{equation}
        \exists m, \exists M\in\R;\ \bsqu{0<m\leq M\ \land \ \pare{
            \forall x\in V,\ m\norm{x}_1 \leq \norm{x} \leq M\norm{x}_1
        }}.
    \end{equation}
\end{defi}

\begin{prop}\proplabel{ノルム同値関係}
    ノルムの同値性は同値関係である。
    \begin{proof}
        ノルム$\norm{\dummy}$, $\norm{\dummy}_1$が同値であることを$\norm{\dummy}\sim\norm{\dummy}_1$と記すことにする。
        すなわち,
        \begin{equation}
            \norm{\dummy}\sim\norm{\dummy}_1\ \defarrow\ \exists m, \exists M\in\R;\ \bsqu{0<m\leq M\ \land \ \pare{
                \forall x\in V,\ m\norm{x}_1 \leq \norm{x} \leq M\norm{x}_1
            }}.
        \end{equation}
        このとき関係$\sim$が反射律,対称律,推移律を満たすことを示す。
        \begin{itemize}
            \item (反射律)
            任意の$\norm{\dummy}$に対し,$\norm{\dummy}\sim\norm{\dummy}$が成立することを示す。
            実際,$m=M=1$とすれば,
            \begin{equation}
                0<m=1\leq M=1\ \land \pare{\forall x\in V,\ 1\cdot \norm{x}\leq \norm{x}\leq 1\cdot \norm{x}}
            \end{equation}
            であるから成立。

            \item (対称律)
            $\norm{\dummy}\sim\norm{\dummy}_1$,すなわち
            \begin{equation}
                \exists m, \exists M\in\R;\ \bsqu{0<m\leq M\ \land \ \pare{
                \forall x\in V,\ m\norm{x}_1 \leq \norm{x} \leq M\norm{x}_1
                }}
            \end{equation}
            のとき,$\norm{\dummy}_1\sim\norm{\dummy}$が成立することを示す。
            後半の不等式を書き直すと,$\frac{1}{M}\norm{x}\leq \norm{x}_1\leq \frac{1}{m}\norm{x}$であり,$0<m\leq M$であるから$0<\frac{1}{M}\leq \frac{1}{m}$であるから,$m' \coloneqq \frac{1}{M}$, $M' \coloneqq  \frac{1}{m}$と取れば確かに$\norm{\dummy}_1\sim\norm{\dummy}$の成立が判る。

            \item (推移律)
            3つのノルム$\norm{\dummy}$, $\norm{\dummy}_1$, $\norm{\dummy}_2$に対し,$\norm{\dummy}\sim \norm{\dummy}_1$と$\norm{\dummy}_1\sim \norm{\dummy}_2$のとき,$\norm{\dummy}\sim \norm{\dummy}_2$が成立することを示す。
            仮定から,
            \begin{align}
                &\exists m_1, \exists M_2\in\R;\ \bsqu{0<m_1\leq M_2\ \land \ \pare{
                \forall x\in V,\ m_1\norm{x}_1 \leq \norm{x} \leq M_1\norm{x}_1
                }},
                \\
                &\exists m_2, \exists M_2\in\R;\ \bsqu{0<m_2\leq M_2\ \land \ \pare{
                \forall x\in V,\ m_2\norm{x}_2 \leq \norm{x}_1 \leq M_2\norm{x}_2
                }}
            \end{align}
            の両方が成立しているから,$\norm{x}\leq M_1\norm{x}_1\leq M_1M_2\norm{x}_2$及び,$\norm{x}\geq m_1\norm{x}_1\geq m_1m_2 \norm{x}_2$より,$m_1m_2\norm{x}_2\leq \norm{x}\leq M_1M_2\norm{x}_2$が従う。
            これより,$m\coloneqq m_1m_2$, $M\coloneqq M_1M_2$と取れば$0<m_1m_2\leq M_1M_2$も成立するので$\norm{\dummy}\sim \norm{\dummy}_2$である。
        \end{itemize}

        以上により,ノルムの同値性は同値関係である。
    \end{proof}
\end{prop}

\begin{thm}\thmlabel{ノルム同値性}
    ベクトル空間$V$が有限次元であれば,$V$上の任意の2つのノルムは同値である。
    \begin{proof}
        $V$を有限次元ベクトル空間とし,$n\coloneqq \dim{V}$とする。
        $V$の基底として,$\seq{\ele_k}{0\leq k\leq n-1}$を取り固定する。
        $V$の元$x$を,$x =\sum_{k=0}^{n-1} x_k \ele_k$と表したときの成分$\seq{x_k}{0\leq k\leq n-1}$を用いて,
        \begin{equation}
            \norm{\dummy}\colon V\to \R;\ x\mapsto \norm{x}\coloneqq \max_{0\leq k\leq n-1}\bra{\abs{x_k}}
            \equlabel{ノルムmax}
        \end{equation}
        と定めると$\norm{\dummy}$はノルムになる。
        以下$\norm{\dummy}$がノルムであることを確かめる:
        \begin{itemize}
            \item (正値性)
            どの$k$についても$0\leq \abs{x_k}$であり,$\displaystyle\norm{x}= \max_k\bra{\abs{x_k}}\geq 0$であるから成立。

            \item (一意性)
            $x=0$のとき,どの$k$についても$x_k=0$であるから$\displaystyle\norm{x}= \max_k\bra{\abs{x_k}}=0$である。
            また,$\norm{x}=0$のとき,任意の$k$に対し,絶対値の非負性から$0\leq \abs{x_k}$であって,$\abs{x_k}\leq {x}=0$であるから$\abs{x_k}=0$である。
            これより$x=0$となる。

            \item (同次性)
            絶対値の非負性と同次性から,$\displaystyle\norm{cx}=\max_k \bra{\abs{cx}}=\abs{c}\cdot \max_k\bra{\abs{x_k}}=\abs{c}\norm{x}$が従う。

            \item (三角不等式)
            絶対値の三角不等式から,各$k$に対して$\abs{x_k+y_k}\leq \abs{x_k}+\abs{y_k}$であるから最大値に関してもこの不等号が成り立つので$\norm{x+y}\leq \norm{x}+\norm{y}$が従う。
        \end{itemize}
        以上により\equref{ノルムmax}で定められた$\norm{\dummy}$はノルムである。

        次に,$V$上の勝手なノルム$\norm{\dummy}_1$を取ってきたときに,\equref{ノルムmax}で定義した$\norm{\dummy}_1$と$\norm{\dummy}$が同値になってしまうことを示す。
        $V$のコンパクト集合$S$を$S\coloneqq \sset{y\in V}{\norm{y}=1}$
        によって定め,関数$f\colon S\to \R$を$\fun{f}{y}\coloneqq \norm{y}_1$と定める。
        $f$の連続性(\corref{ノルム連続性})と$S$がコンパクト集合であることから,$f$の値域には最小元$m$と最大元$M$が存在する。
        $S$上で$y\ne {0}$であり,ノルムの正値性から$0<m\leq M$である。
        特に,$\norm{y}=1$ならば$y\in S$であり,$m$, $M$の定義から$m\leq \fun{f}{y}\leq M$及び$\fun{f}{y}=\norm{y}_1$なので$m\leq \norm{x}_1\leq M$が従う。
        ここで,$V$上の一般の$x\ne 0$に対して$y\coloneqq x / \norm{x}$とすると$\norm{y}=\norm{x/\norm{x}}=\norm{x}/\norm{x}=1$より$y\in S$であるから$\fun{f}{y}=\norm{x/\norm{x}}_1=\norm{x}_1/\norm{x}$であって,$m\leq \norm{x}_1/\norm{x}\leq M$より$m\norm{x}\leq \norm{x}_1\leq M\norm{x}$である。
        $x=0$についても$\norm{0}=\norm{0}_1=0$であって,$m \norm{0}_1\leq \norm{0}\leq M\norm{0}_1$は成立するので,$\norm{\dummy}_1$と$\norm{\dummy}$は同値である。

        有限次元ベクトル空間$V$上で与えられた任意の2つのノルム$\norm{\dummy}_1$, $\norm{\dummy}_2$はそれぞれ\equref{ノルムmax}で定義された$\norm{\dummy}$と同値であり,ノルムの同値は同値関係(\propref{ノルム同値関係})なので,$\norm{\dummy}_1$と$\norm{\dummy}_2$も同値である。
    \end{proof}
\end{thm}

有限次元複素ベクトル空間$\CC^n$上のノルムの例を挙げる。
以下では,$x=\pare{x_0,x_2,\dots,x_{n-1}}$の各成分を複素数とし,$\abs{\dummy}$を複素数の絶対値とする。

\begin{eg}[Euclidノルム]\eglabel{ノルムEuclid}
    標準的な距離:
    \begin{equation}
        \norm{x}_2\coloneqq \pare{\sum_{k=0}^{n-1} \abs{x_k}^2}^{\frac{1}{2}}
    \end{equation}
    はノルムである。
\end{eg}

\begin{eg}[一様ノルム]\eglabel{ノルム一様}
    \thmref{ノルム同値性}の証明において\equref{ノルムmax}:
    \begin{equation}
        \norm{x}_\infty \coloneqq \max_{0\leq k\leq n-1}\bra{\abs{x_k}}
        \tag{\ref{equ:ノルムmax}}
    \end{equation}
    はノルムである。
    後述するように,$p$ノルムにおいて$p\to\infty$の極限で再現されることから,$\infty$ノルムとも呼ぶ。
\end{eg}

\begin{eg}[$p$ノルム]\eglabel{ノルムpノルム}
    $p$を$1$以上の実数とする。
    このとき,
    \begin{equation}
        \norm{x}_p \coloneqq \pare{\sum_{k=0}^{n-1} \abs{x_k}^p}^{\frac{1}{p}}
    \end{equation}
    はノルムである。
    $1$ノルムのことをManhattan距離と呼ぶ。
    $2$ノルムはEuclidノルムに一致する。
    特に,$p\to\infty$の極限で$p$ノルムは\egref{ノルム一様}の一様ノルムに一致する。
\end{eg}

以降,有限次元ベクトル空間のノルムは\egref{ノルム一様}の一様ノルムであるとする。

\begin{lem}\lemlabel{ノルム積分の不等式}
    $I$を$\R$上の閉区間であるとする。
    $f\colon I\to \R^n$を連続なベクトル値関数とするとき,
    \begin{equation}
        \norm{\int_I dt\, \fun{f}{t}}\leq \int_I dt\, \norm{\fun{f}{t}}
    \end{equation}
    が成立する。
    \begin{proof}
        $I$上可積分な$g$と絶対値に関する以下の不等式:
        \begin{equation}
            \abs{\int_I dt\, \fun{g}{t}}\leq \int_I dt\, \abs{\fun{g}{t}}
        \end{equation}
        を用いる。
        \begin{equation}
            \forall j\in \N,\, \forall t\in I,\ \bsqu{
                0\leq j\leq n-1\ \Longrightarrow\ \fun{f_j}{t}\leq \abs{\fun{f_j}{t}}\leq \norm{\fun{f}{t}}
            }
        \end{equation}
        であるから,
        \begin{equation}
            \forall j\in \N,\, \squ{
                0\leq j\leq n-1\ \Longrightarrow\ 
                \int_I dt\, \fun{f_j}{t}\leq \int_I dt\, \abs{\fun{f_j}{t}}
                \leq \int_I dt\, \norm{\fun{f}{t}}
            }
        \end{equation}
        である。
        ここで,
        \begin{equation}
            \abs{\int_I dt\, \fun{f_j}{t}}\leq \int_I dt\, \abs{\fun{f_j}{t}}\leq \int_I dt\, \norm{\fun{f}{t}}
        \end{equation}
        であるから,
        \begin{equation}
            \norm{\int_I dt \, \fun{f}{t}}
            =\max_{0\leq j\leq n-1}\bra{\abs{\int_I dt\, \fun{f_j}{t}}}\leq \int_I dt\, \norm{\fun{f}{t}}
        \end{equation}
        が成立する。
    \end{proof}
\end{lem}

\begin{lem}
    $S$, $T$を適当な集合とし,写像$f\colon S\times T\to \R$を与える。
    このとき,
    \begin{equation}
        \sup_{a\in S}\bra{\max_{b\in T}\bra{\fun{f}{a,b}}}
        =\max_{b\in T}\bra{\sup_{x\in a}\bra{\fun{f}{a,b}}}
    \end{equation}
    が成立する。
    \begin{proof}
        \begin{prooftree}
            {
                close with={\times},
                to prove={
                    \pare{\pare{\forall a} \pare{p(a) \to \pare{\pare{\forall b} \pare{ q(b) \to r(a,b)}}}}
                    \vdash
                    \pare{\pare{\forall b} \pare{q(b) \to \pare{\pare{\forall a} \pare{ p(a) \to r(a,b)}}}},
                }
            }
            [
                \pare{\forall a} \pare{p(a) \to \pare{\pare{\forall b} \pare{ q(b) \to r(a,b)}}}, just={Assumption}, name={1}
                    [\lnot\pare{\pare{\forall b} \pare{q(b) \to \pare{\pare{\forall a} \pare{ p(a) \to r(a,b)}}}}, just={$\lnot$ Conclusion}
                    [\lnot\pare{q(b')\to\pare{\pare{\forall a}\pare{p(a)\to r(a,b')}}}, just={$\lnot\pare{\forall b}$ {}:!u}
                        [q(b'), just={$\lnot \to$ {}:!u}, name={4}
                            [\lnot\pare{\pare{\forall a}\pare{p(a)\to r(a,b')}}, just={$\lnot \to$ {}:!uu}
                                [\lnot\pare{p(a')\to r(a',b')}, just={$\lnot\pare{\forall a}$ {}:!u}
                                    [p(a'), just={$\lnot \to$ {}:!u}, name={7}
                                        [{\lnot r(a',b')}, just={$\lnot\to$ {}:!uu}, name={8}
                                            [p(a') \to \pare{\pare{\forall b}\pare{q(b)\to r(a',b)}}, just={$\pare{\forall a}$ {}:1} 
                                                [\lnot p(a'), just={$\to$ {}:!u}, close={:7,!c}]
                                                [\pare{\forall b}\pare{q(b)\to r(a',b)}
                                                [{q(b')\to r(a',b')}, just={$\pare{\forall b}$ {}:!u}
                                                    [\lnot q(b'), , just={$\to$:!u},close={:4,!c}]
                                                    [{r(a',b')}, close={:8,!c}]
                                                ]
                                                ]
            ]]]]]]]]]
        \end{prooftree}
    \end{proof}
\end{lem}



\section{Lipschitz条件}


\section{Picardの逐次近似法}


\end{document}
