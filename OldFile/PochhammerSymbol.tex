\documentclass[a4paper,draft]{ltjsarticle}
\usepackage{preamble}

\begin{document}
%\maketitle

\section{Pochhammer記号}\label{sec:poch}
この章ではPochhammer記号についての公式をまとめる。
出典の多くは\cite{nkswtr}による。


\begin{defi}[Pochhammer記号]
    $a$を複素数,$n$を整数とする。
    Pochhammer記号$\poch{a}{n}$とは,Gamma関数$\fun{\Gamma}{z}$を用いて,
    \begin{equation}
        \poch{a}{n}\coloneqq \dfrac{\fun{\Gamma}{a+n}}{\fun{\Gamma}{a}}
    \end{equation}
    と定義されるものである。
\end{defi}

\begin{defi}[Pochhammer記号の積]
    (この章では直接用いることはないが,)Pochhammer記号の積を,
    \begin{equation}
        \poch{a_1,a_2,\dots,a_m}{n}\coloneqq \prod_{k=1}^m\poch{a_k}{n}=\frac{\fun{\Gamma}{a_1+n}\fun{\Gamma}{a_2+n}\cdots\fun{\Gamma}{a_m+n}}{\fun{\Gamma}{a_1}\fun{\Gamma}{a_2}\cdots\fun{\Gamma}{a_m}}
    \end{equation}
    と表記することがある。
\end{defi}

Gamma関数を用いた定義をすれば,$n$は整数である必要はないが,実用上はほとんど整数に限ってよい。

以降,特に断りがなければ$a$, $b$を複素数,$n$, $k$を自然数とする。



\begin{prop}
    \begin{equation}
        \poch{1}{n}=n!
    \end{equation}
    \begin{proof}
        $n$を自然数としたとき,$\fun{\Gamma}{n+1}=n!$に注意すれば,
        \begin{equation}
            \poch{1}{n}=\frac{\fun{\Gamma}{1+n}}{\fun{\Gamma}{1}}=\frac{n!}{0!}=n!
        \end{equation}
        となる。
    \end{proof}
\end{prop}

\begin{prop}\label{prop:poch-(-m)_{n}=0}
    $m$を自然数とし,$n$は$m+1\leq n$を満たす自然数であるとすると,
    \begin{equation}
        \poch{-m}{n}=0
    \end{equation}
    である。
    \begin{proof}
        $m$を自然数としたとき,$\fun{\Gamma}{-m}$は1位の極になっていることを用いる。
        $-m+n\geq 1$だから$\fun{\Gamma}{-m+n}$は有限の値を取るので,
        \begin{align}
            \poch{-m}{n}=\frac{\fun{\Gamma}{-m+n}}{\fun{\Gamma}{-m}}=0
        \end{align}
        である。
    \end{proof}
\end{prop}

命題\ref{prop:poch-(-m)_{n}=0}は,超幾何級数が有限次の多項式になるための条件を導く際に用いる。
これは,偏微分方程式を変数分離した際に出てくる定数の条件を,解が発散しないための条件(超幾何級数が有限次で切れる)から決める際にも用いる\footnote{例えば,水素原子のSchr\"{o}dinger方程式の$\theta$方向の方程式(Legendreの陪微分方程式)が,$\theta=0$, $\pi$で発散しないための条件}。

\begin{prop}\label{prop:poch-an+b}
    \begin{align}
        n+a&=a\cdot \frac{\poch{a+1}{n}}{\poch{a}{n}}
        \\
        an+b&=b\cdot \frac{\poch{\frac{b}{a}+1}{n}}{\poch{\frac{b}{a}}{n}} 
    \end{align}
    \begin{proof}
        1つ目は,
        \begin{align}
            n+a&=\pare{n+a}\cdot\frac{\fun{\Gamma}{n+a}}{\fun{\Gamma}{n+a}}
            =\Poch{a+1}{n}\cdot \frac{\fun{\Gamma}{a}}{\fun{\Gamma}{n+a}}\cdot\frac{\fun{\Gamma}{a+1}}{\fun{\Gamma}{a}}
            \\
            &=\poch{a+1}{n}\cdot\frac{1}{\poch{a}{n}}\cdot a=a\cdot \frac{\poch{a+1}{n}}{\poch{a}{n}}
        \end{align}
        とすればよい。

        2つ目は,直前に示した式を用いて,
        \begin{equation}
            an+b=a\pare{n+\frac{b}{a}}
            =a\cdot \frac{b}{a}\cdot \frac{\poch{\frac{b}{a}+1}{n}}{\poch{\frac{b}{a}}{n}}=b\cdot \frac{\poch{\frac{b}{a}+1}{n}}{\poch{\frac{b}{a}}{n}} 
        \end{equation}
        である。
    \end{proof}
\end{prop}

命題\ref{prop:poch-an+b}は,Maclaurin級数が既知であるような冪級数を超幾何級数表示に直す際に,最も頻繁に用いる公式である。

\begin{prop}\label{prop:poch-(a)_n=prod(a+k)}
    \begin{align}
        \poch{a}{0}&=1
        \\
        \poch{a}{n}&=\prod_{k=0}^{n-1}\pare{a+k}\quad\pare{n \geq 1}
        \\
        \poch{a}{-n}&=\prod_{k=1}^{n}\frac{1}{a-k}\quad\pare{n\geq 1}
    \end{align}
    \begin{proof}
        1つ目は定義から,
        \begin{equation}
            \poch{a}{0}=\frac{\fun{\Gamma}{a+0}}{\fun{\Gamma}{a}}=\frac{\fun{\Gamma}{a}}{\fun{\Gamma}{a}}=1
        \end{equation}
        となる。

        2つ目は,分子に対して$z\fun{\Gamma}{z}=\fun{\Gamma}{z+1}$を繰り返し用いれば,
        \begin{align}
            \poch{a}{n}&=\frac{\fun{\Gamma}{a+n}}{\fun{\Gamma}{a}}=\frac{\pare{a+n-1}\fun{\Gamma}{a+n-1}}{\fun{\Gamma}{a}}=\frac{\pare{a+n-1}\pare{a+n-2}\fun{\Gamma}{a+n-2}}{\fun{\Gamma}{a}}=\cdots
            \\
            &=\frac{\pare{a+n-1}\pare{a+n-2}\cdots\pare{a+1}a\fun{\Gamma}{a}}{\fun{\Gamma}{a}}=\pare{a+n-1}\pare{a+n-2}\cdots\pare{a+1}a
            \\
            &=\prod_{k=0}^{n-1}\pare{a+k}
        \end{align}
        となる。

        3つ目は分母に対して同じように,
        \begin{align}
            \poch{a}{-n}&=\frac{\fun{\Gamma}{a-n}}{\fun{\Gamma}{a}}=\frac{\fun{\Gamma}{a-n}}{\pare{a-1}\fun{\Gamma}{a-1}}=\frac{\fun{\Gamma}{a-n}}{\pare{a-1}\pare{a-2}\fun{\Gamma}{a-2}}=\cdots
            \\
            &=\frac{\fun{\Gamma}{a-n}}{\pare{a-1}\pare{a-2}\cdots\pare{a-n}\fun{\Gamma}{a-n}}=\frac{1}{\pare{a-1}\pare{a-2}\cdots\pare{a-n}}
            \\
            &=\prod_{k=1}^n\frac{1}{a-k}
        \end{align}
    \end{proof}
\end{prop}

超幾何級数の定数項が必ず$1$であるという事実は,$\poch{a}{0}=1$から出てくる。

\begin{prop}
    \begin{align}
        \poch{a}{n}&=\pare{-1}^n\poch{1-n-a}{n}
        \\
        \poch{a}{-n}&=\dfrac{\pare{-1}^n}{\poch{1-a}{n}} \label{eq:Poch-poch{a}{-n}}
    \end{align}
    \begin{proof}
        Gamma関数の相反公式(定理\ref{thm:Gamma-相反公式}):
        \begin{equation}
            \fun{\Gamma}{a}\fun{\Gamma}{1-a}=\frac{\pi}{\fun{\sin}{\pi a}}    
        \end{equation}
        を用いる。
        三角関数の加法定理などから,$\fun{\sin}{\pi\pare{a+n}}=\pare{-1}^n\fun{\sin}{\pi a}$が成立することに注意して,
        \begin{align}
            \pare{-1}^n\poch{1-n-a}{n}
            &=\pare{-1}^n\cdot\frac{\fun{\Gamma}{1-a}}{\fun{\Gamma}{1-n-a}}\cdot \frac{\fun{\Gamma}{a}}{\fun{\Gamma}{n+a}}\cdot\frac{\fun{\Gamma}{n+a}}{\fun{\Gamma}{a}}
            \\
            &=\frac{\fun{\Gamma}{n+a}}{\fun{\Gamma}{a}}\cdot \pare{-1}^n \cdot \fun{\Gamma}{a}\fun{\Gamma}{1-a}\cdot\frac{1}{\fun{\Gamma}{n+a}\fun{\Gamma}{1-\spare{n+a}}}
            \\
            &=\poch{a}{n}\cdot\pare{-1}^n\cdot \frac{\pi}{\fun{\sin}{\pi a}}\cdot\frac{\fun{\sin}{\pi\pare{n+a}}}{\pi}
            \\
            &=\poch{a}{n}\cdot\pare{-1}^n\cdot \frac{\pi}{\fun{\sin}{\pi a}}\cdot\frac{\pare{-1}^n\fun{\sin}{\pi a}}{\pi}
            \\
            &=\poch{a}{n}
        \end{align}

        2つ目も同様に,相反公式を用いれば,
        \begin{align}
            \poch{a}{-n}&=\frac{\fun{\Gamma}{a-n}}{\fun{\Gamma}{a}}\cdot\frac{\fun{\Gamma}{1-a+n}}{\fun{\Gamma}{1-a}}\cdot\frac{\fun{\Gamma}{1-a}}{\fun{\Gamma}{1-a+n}}
            \\
            &=\frac{\fun{\Gamma}{1-a}}{\fun{\Gamma}{1-a+n}}
            \cdot \fun{\Gamma}{a-n}\fun{\Gamma}{1-\spare{a-n}}
            \cdot\frac{1}{\fun{\Gamma}{a}\fun{\Gamma}{1-a}}
            \\
            &=\frac{1}{\poch{1-a}{n}}
            \cdot \frac{\pi}{\fun{\sin}{\pi\pare{a-n}}}\cdot \frac{\fun{\sin}{\pi a}}{\pi}
            \\
            &=\frac{1}{\poch{1-a}{n}}\cdot \frac{\pi}{\pare{-1}^n\fun{\sin}{\pi a}}\cdot \frac{\fun{\sin}{\pi a}}{\pi}
            \\
            &=\frac{\pare{-1}^n}{\poch{1-a}{n}}
        \end{align}
    \end{proof}
\end{prop}

\begin{prop}\label{prop:poch-(a)_{n+k}}
    \begin{align}
        \poch{a}{n+k}&=\poch{a}{n}\poch{a+n}{k}
        \\
        \poch{a}{n-k}&=\dfrac{\pare{-1}^k\poch{a}{n}}{\poch{1-n-a}{k}}
        \\
        \poch{-n}{k}&=\dfrac{\pare{-1}^k n!}{\pare{n-k}!} \label{eq:Poch-poch(-n)_{-k}}
    \end{align}
    \begin{proof}
        1つ目は,
        \begin{equation}
            \poch{a}{n+k}=\Poch{a}{n+k}=\Poch{a}{n}\cdot\Poch{a+n}{k}=\poch{a}{n}\poch{a+n}{k}
        \end{equation}

        2つ目は直前に示した式と\eqref{eq:Poch-poch{a}{-n}}式から,
        \begin{equation}
            \poch{a}{n-k}=\poch{a}{n}\poch{a+n}{-k}=\poch{a}{n}\cdot\frac{\pare{-1}^k}{\poch{1-\spare{a+n}}{k}}=\dfrac{\pare{-1}^k\poch{a}{n}}{\poch{1-n-a}{k}}
        \end{equation}

        3つ目は\eqref{eq:Poch-poch{a}{-n}}式と,$\fun{\Gamma}{m+1}=m!$から,
        \begin{equation}
            \poch{-n}{k}=\frac{\pare{-1}^{-k}}{\poch{1+n}{-k}}=\pare{-1}^k\cdot \frac{\fun{\Gamma}{1+n}}{\fun{\Gamma}{1+n-k}}=\dfrac{\pare{-1}^k n!}{\pare{n-k}!}
        \end{equation}
    \end{proof}
\end{prop}

\begin{cor}\label{cor:poch-(a)_{n+1}}
    \begin{equation}
        \poch{a}{n+1}=\pare{a+n}\poch{a}{n}
    \end{equation}
    \begin{proof}
        \begin{equation}
            \poch{a}{n+1}=\poch{a}{n}\poch{a+n}{1}=\poch{a}{n}\Poch{a+n}{1}
            =\poch{a}{n}\frac{\pare{a+n}\fun{\Gamma}{a+n}}{\fun{\Gamma}{a+n}}=\pare{a+n}\poch{a}{n}
        \end{equation}
    \end{proof}
\end{cor}

\begin{prop}
    \begin{equation}
        \binom{n}{k}=\pare{-1}^k\frac{\poch{-n}{k}}{k!}
    \end{equation}
    \begin{proof}
        \eqref{eq:Poch-poch(-n)_{-k}}式から,
        \begin{equation}
            \poch{-n}{k}=\frac{\pare{-1}^k n!}{\pare{n-k}!}\cdot \frac{k!}{k!}=\pare{-1}^k k!\binom{n}{k}
        \end{equation}
        となるので,両辺に$\pare{-1}^k/k!$を掛けると,
        \begin{equation}
            \binom{n}{k}=\pare{-1}^k\frac{\poch{-n}{k}}{k!}
        \end{equation}
    \end{proof}
\end{prop}

\begin{prop}
    \begin{equation}
        \poch{a}{2n}=2^{2n}\poch{\frac{a}{2}}{n}\poch{\frac{a+1}{2}}{n}
    \end{equation}
    \begin{proof}
        Gamma関数に関するLegendreの倍角公式(定理\ref{thm-Gamma-Legendreの倍角公式})から得られる,
        \begin{equation}
            \fun{\Gamma}{a}=\fun{\Gamma}{2\cdot\frac{a}{2}}=\frac{2^{a-1}}{\sqrt{\pi}}\fun{\Gamma}{\frac{a}{2}}\fun{\Gamma}{\frac{a+1}{2}}
        \end{equation}
        \begin{equation}
            \fun{\Gamma}{a+2n}=\fun{\Gamma}{2\pare{\frac{a}{2}+n}}=\frac{2^{a+2n-1}}{\sqrt{\pi}}\fun{\Gamma}{\frac{a}{2}+n}\fun{\Gamma}{\frac{a+1}{2}+n}
        \end{equation}
        を用いることで,
        \begin{align}
            \poch{a}{2n}&=\Poch{a+2n}{a}=\frac{\frac{2^{a+2n-1}}{\sqrt{\pi}}\fun{\Gamma}{\frac{a}{2}+n}\fun{\Gamma}{\frac{a+1}{2}+n}}{\frac{2^{a-1}}{\sqrt{\pi}}\fun{\Gamma}{\frac{a}{2}}\fun{\Gamma}{\frac{a+1}{2}}}
            \\
            &=2^{2n}\Poch{\frac{a}{2}}{n}\Poch{\frac{a+1}{2}}{n}=2^{2n}\poch{\frac{a}{2}}{n}\poch{\frac{a+1}{2}}{n}
        \end{align}
    \end{proof}
\end{prop}

\begin{prop}\label{prop:poch-(2n)!}
    \begin{align}
        \pare{2n}!&=2^{2n}\poch{1}{n}\poch{\frac{1}{2}}{n} \label{eq:poch-(2n)!}
        \\
        \poch{\frac{1}{2}}{n}&=\dfrac{\pare{2n}!}{2^{2n}n!} \label{eq:poch-(1/2)_{n}}
    \end{align}
    \begin{proof}
        Gamma関数に関するLegendreの倍角公式(定理\ref{thm-Gamma-Legendreの倍角公式}),$\fun{\Gamma}{\frac{1}{2}}=\sqrt{\pi}$,$\poch{1}{n}=n!$から,
        \begin{align}
            \pare{2n}!&=\fun{\Gamma}{2n+1}=2n\fun{\Gamma}{2n}=2n\cdot \frac{2^{2n-1}}{\sqrt{\pi}}\fun{\Gamma}{n}\fun{\Gamma}{n+\frac{1}{2}}
            \\
            &=2^{2n}\cdot n\fun{\Gamma}{n}\cdot \frac{\fun{\Gamma}{n+\frac{1}{2}}}{\fun{\Gamma}{\frac{1}{2}}}=2^{2n}\cdot n!\cdot \frac{\fun{\Gamma}{n+\frac{1}{2}}}{\fun{\Gamma}{\frac{1}{2}}}
            \\
            &=2^{2n}\poch{1}{n}\poch{\frac{1}{2}}{n}
        \end{align}

        後者は今得られた式を書き換えて,
        \begin{equation}
            \poch{\frac{1}{2}}{n}=\frac{\pare{2n}!}{2^{2n}\poch{1}{n}}=\frac{\pare{2n}!}{2^{2n}n!}
        \end{equation}
    \end{proof}
\end{prop}




\begin{prop}\label{prop:poch-二重階乗}
    \begin{equation}
        \frac{1}{n!}\poch{\frac{1}{2}}{n}=\frac{1}{2^{2n}}\binom{2n}{n}=\frac{\pare{2n-1}!!}{\pare{2n}!!} \label{eq:poch-(2n-1)!!/(2n)!!}
    \end{equation}
    \begin{proof}
        最初の等号は,\eqref{eq:poch-(1/2)_{n}}式に帰着させて,
        \begin{equation}
            \frac{1}{2^{2n}}\binom{2n}{n}=\frac{1}{2^{2n}}\frac{\pare{2n}!}{n!^2}=\frac{1}{n!}\cdot\dfrac{\pare{2n}!}{2^{2n}n!}=\frac{1}{n!}\poch{\frac{1}{2}}{n}
        \end{equation}
        から成立。

        次の等号を示す前に,二重階乗の定義を確認しておく。
        偶数の二重階乗は,
        \begin{align}
            \pare{2n}!!&\coloneqq 2n\cdot \pare{2n-2}\cdot\pare{2n-4}\cdots 4\cdot 2
            \\
            &=2^n\cdot n\cdot \pare{n-1}\cdot \pare{n-2}\cdots 2\cdot 1=2^n n!
        \end{align}
        で定義されており,奇数の二重階乗は,
        \begin{align}
            \pare{2n-1}!!&\coloneqq \pare{2n-1}\pare{2n-3}\cdots 3\cdot 1
            \\
            &=\frac{2n\cdot \pare{2n-1}\cdot \pare{2n-2}\cdot\pare{2n-3}\cdots 3\cdot 2\cdot 1}{ 2n\cdot \pare{2n-2}\cdot\pare{2n-4}\cdots 4\cdot 2}
            \\
            &=\frac{\pare{2n}!}{\pare{2n}!!}=\frac{\pare{2n}!}{2^n n!}
        \end{align}
        である。

        二重階乗を階乗で表した式を用いて最右辺を計算すると,
        \begin{equation}
            \frac{\pare{2n-1}!!}{\pare{2n}!!}=\frac{\pare{2n}!}{2^n n!}\cdot \frac{1}{2^n n!}=\frac{1}{2^{2n}}\frac{\pare{2n}!}{n!^2}=\frac{1}{2^{2n}}\binom{2n}{n}
        \end{equation}
        となる。
        よって,
        \begin{equation}
            \frac{1}{n!}\poch{\frac{1}{2}}{n}=\frac{1}{2^{2n}}\binom{2n}{n}=\frac{\pare{2n-1}!!}{\pare{2n}!!}
        \end{equation}
        が成立する。
    \end{proof}
\end{prop}

最後に示した\eqref{eq:poch-(2n-1)!!/(2n)!!}式は,冪級数展開の係数に二重階乗を含むような関数\footnote{完全楕円積分$\fun{K}{x}$,第二種完全楕円積分$\fun{E}{x}$,逆正弦関数$\fun{\arcsin}{x}$など}を超幾何級数表示に直す際に用いる。



\end{document}