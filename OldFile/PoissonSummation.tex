\documentclass[a4paper,draft]{ltjsarticle}
\usepackage{preamble}
\begin{document}
\section{Poisson和公式}\label{sec:Poisson}
Poisson和公式は,元の関数の整数点における値の無限和と,Fourier変換した後の関数の整数点における値の無限和を結びつける公式で,様々な関数等式を導くことのできる大変強力な公式である。
また,Poisson和公式から導かれるTheta変換公式は楕円関数論や数論においても重要な役割を果たすが,ここではその詳細については立ち入らず,証明の紹介だけする。

本章では,正弦関数の無限積表示(定理\ref{thm:Poisson-sin})の証明を目標にする。

\subsection{Fourier変換とPoisson和公式}
\begin{defi}[Fourier変換]
    関数$\fun{f}{x}$のFourier変換$\fun{\tilde{f}}{\xi}$を,
    \begin{equation}
        \fun{\tilde{f}}{\xi}\coloneqq \int_{-\infty}^\infty d\xi\ \fun{f}{x}e^{-2\pi i \xi x}
    \end{equation}
    で定義する。
    但し,収束性のために$\fun{f}{x}$は以下の2条件を満たすとする:
    \begin{enumerate}[label=(\roman*)]
        \item $\exists \delta>0,\ \fun{f}{x}=\fun{\mathcal{O}}{\abs{x}^{-\delta -1}}\ \pare{x\to\pm\infty}$
        \item $\exists \varepsilon>0,\ \fun{\tilde{f}}{\xi}=\fun{\mathcal{O}}{\abs{\xi}^{-\varepsilon-1}}\ \pare{\xi\to\pm\infty}$
    \end{enumerate}
\end{defi}

\begin{thm}[Poisson和公式]
    Fourier変換可能な関数$f$について,以下が成立する:
    \begin{equation}
        \sum_{m=-\infty}^\infty \fun{f}{x+m} = \sum_{n=-\infty}^\infty \fun{\tilde{f}}{n}e^{-2n\pi i x}
    \end{equation}
    これを,(一般化された)Poisson和公式と呼ぶ。

    \begin{proof}
        関数$g$を以下のように定める:
        \begin{equation}
            \fun{g}{x}\coloneqq \sum_{m=-\infty}^\infty \fun{f}{x+m}
        \end{equation}
        このとき,任意の$x$について$\fun{g}{x+1}=\fun{g}{x}$が成立するから,関数$g$は周期$1$の周期関数であるから,複素Fourier級数展開:
        \begin{equation}
            \fun{g}{x}=\sum_{n=-\infty}^\infty c_n e^{2n\pi i x}
        \end{equation}
        を持つ。
        係数$c_n$は,
        \begin{equation}
            c_n=\int_0^1 dx\ \fun{g}{x} e^{-2\pi i nx}
            = \sum_{m=-\infty}^\infty \int_0^1 dx\ \fun{f}{x+m}e^{-2\pi i nx}
        \end{equation}
        である。
        各$m$について,$x+m\mapsto x'$とすると,区間は$\squ{0,1}\mapsto\squ{m,m+1}$となり,複素指数関数の周期性から,$e^{-2\pi i \spare{x'-m}}=e^{-2\pi i nx'}$である。
        $m$について和を取ると,
        \begin{align}
            c_n
            &=\sum_{m=-\infty}^\infty \int_0^1 dx\ \fun{f}{x+m}e^{-2 \pi i n x}
            \\
            &=\sum_{m=-\infty}^\infty \int_m^{m+1} dx'\ \fun{f}{x'}e^{-2\pi i nx'}
            \\
            &=\int_{-\infty}^\infty dx\ \fun{f}{x}e^{-2\pi i n x}
            \\
            &=\fun{\tilde{f}}{n}
        \end{align}
        となる。
        これより,
        \begin{equation}
            \fun{g}{x}=\sum_{m=-\infty}^\infty \fun{f}{x+m}=\sum_{n=-\infty}^\infty \fun{\tilde{f}}{n}e^{-2\pi i nx}
        \end{equation}
        を得る。
    \end{proof}
\end{thm}

\begin{cor}[Poisson和公式]
    Poisson和公式において,$x=0$とすれば,
    \begin{equation}
        \sum_{m=-\infty}^\infty \fun{f}{m}=\sum_{n=-\infty}^\infty \fun{\tilde{f}}{n}
    \end{equation}
    となる。
    この式をPoisson和公式と呼ぶことが多い。
\end{cor}

\subsection{coth, cosechの級数展開}
\begin{eg}[$\fun{\coth}{\pi x}$の無限級数展開]\label{eg:Poisson-coth}
    $\fun{f}{x}\coloneqq e^{-2\pi a\abs{x}}$として,Poisson和公式を適用する。
    $f$のFourier変換は,
    \begin{align}
        \fun{\tilde{f}}{\xi}&=\int_{-\infty}^\infty dx\ e^{-2\pi a\abs{x}}e^{2\pi i \xi x}
        \\
        &=\int_{-\infty}^0 dx\ e^{2\pi\spare{a-i\xi}x}
        +\int_0^\infty dx\ e^{-2\pi\spare{a+i\xi}x}
        \\
        &=\dfrac{1}{2\pi}\pare{\dfrac{1}{a-i\xi}+\dfrac{1}{a+i\xi}}
        \\
        &=\dfrac{a}{\pi}\cdot\dfrac{1}{\xi^2+a^2}
    \end{align}
    となる。
    Poisson和公式の左辺は,
    \begin{align}
        \sum_{m=-\infty}^\infty \fun{f}{m}
        &=\sum_{m=-\infty}^{-1}e^{2\pi a m}
        + 1
        + \sum_{m=1}^\infty e^{2\pi a m}
        \\
        &= 1 + 2\sum_{m=1}^\infty e^{-2\pi am}
        = 1 + \dfrac{2e^{-2\pi a}}{1-e^{-2\pi a}}
        \\
        &=\dfrac{e^{\pi a}+e^{-\pi a}}{e^{\pi a}-e^{-\pi a}}=\fun{\coth}{\pi a}
    \end{align}
    となるから,右辺と比較して,
    \begin{equation}
        \fun{\coth}{\pi a}=\sum_{m=-\infty}^\infty \dfrac{a}{\pi}\cdot\dfrac{1}{m^2+a^2}
    \end{equation}
    を得る。
    あるいは,
    \begin{equation}
        \sum_{m=1}^\infty \dfrac{1}{m^2+a^2}=\dfrac{\pi}{2a}\fun{\coth}{\pi a}-\dfrac{1}{a^2}
    \end{equation}
    と書き直すこともできる。
\end{eg}

\begin{eg}[$\fun{\cosech}{\pi a}$の無限級数展開]\label{eg:cosech}
    例\ref{eg:Poisson-coth}と同じく$\fun{f}{x}=e^{-2\pi a\abs{x}}$として,一般化されたPoisson和公式で$x=\frac{1}{2}$とすることで,
    \begin{equation}
        \sum_{n=1}^\infty \dfrac{\pare{-1}^{n-1}}{n^2+a^2}
        =\dfrac{1}{2a^2}-\dfrac{\pi}{2a}\fun{\cosech}{\pi a}
    \end{equation}
    を得ることもできる。
\end{eg}

\begin{cor}
    以下が成立する:
    \begin{equation}
        \sum_{n=1}^\infty \dfrac{\pare{-1}^{n-1}}{n^3}\fun{\cosech}{\pi n}=\dfrac{\pi^3}{360}
    \end{equation}
    
    \begin{proof}
        例\ref{eg:cosech}で述べた$\fun{\cosech}{\pi n}$に関する和公式:
        \begin{equation}
            \sum_{n=1}^\infty \dfrac{\pare{-1}^{n-1}}{n^2+m^2}=\dfrac{1}{2m^2}-\dfrac{\pi}{2m}\fun{\cosech}{\pi m}
        \end{equation}
        の両辺を$\pare{-1}^{m-1}m^2$で割って,$m=1,2,\dots$で足し合わせると,左辺は,
        \begin{align}
            &\sum_{m=1}^\infty \dfrac{\pare{-1}^{m-1}}{m^2}\sum_{n=1}^\infty \dfrac{\pare{-1}^{n-1}}{n^2+m^2}
            \\
            &=\sum_{n=1}^\infty\sum_{m=1}^\infty \dfrac{1}{n^2}\pare{\dfrac{1}{m^2}-\dfrac{1}{m^2+n^2}}
            \\
            &=\sum_{n=1}^\infty \dfrac{\pare{-1}^{n-1}}{n^2}\sum_{m=1}^\infty \pare{\dfrac{\pare{-1}^{m-1}}{m^2}-\dfrac{\pare{-1}^{m-1}}{m^2+n^2}}
            \\
            &=\pare{\sum_{n=1}^\infty \dfrac{\pare{-1}^{n-1}}{n^2}}^2-\sum_{n=1}^\infty \dfrac{\pare{-1}^{n-1}}{n^2}\pare{\dfrac{1}{2n^2}-\dfrac{\pi}{2n}\fun{\cosech}{\pi n}}
            \\
            &=\pare{\dfrac{\pi^2}{12}}^2-\dfrac{1}{2}\sum_{n=1}^\infty \dfrac{\pare{-1}^{n-1}}{n^4}+\dfrac{\pi}{2}\sum_{n=1}^\infty \dfrac{\pare{-1}^{n-1}}{n^3}\fun{\cosech}{\pi n}
            \\
            &=\dfrac{\pi^4}{144}-\dfrac{7\pi^4}{1440}
            +\dfrac{\pi}{2}\sum_{n=1}^\infty \dfrac{\pare{-1}^{n-1}}{n^3}\fun{\cosech}{\pi n}
            \\
            &=\dfrac{3\pi^4}{1440}+\dfrac{\pi}{2}\sum_{n=1}^\infty \dfrac{\pare{-1}^{n-1}}{n^3}\fun{\cosech}{\pi n}
        \end{align}
        となり,右辺は,
        \begin{align}
            &\dfrac{1}{2}\sum_{m=1}^\infty \dfrac{\pare{-1}^{m-1}}{m^4}-\dfrac{\pi}{2}\sum_{n=1}^\infty \dfrac{\pare{-1}^{n-1}}{n^3}\fun{\cosech}{\pi n}
            \\
            &=\dfrac{7\pi^4}{1440}
            -\dfrac{\pi}{2}\sum_{n=1}^\infty \dfrac{\pare{-1}^{n-1}}{n^3}\fun{\cosech}{\pi n}
        \end{align}
        となる。
        但し,
        \begin{equation}
            \sum_{n=1}^\infty \frac{\pare{-1}^{n-1}}{n^2}=\frac{\pi^2}{12},\
            \sum_{n=1}^\infty \frac{\pare{-1}^{n-1}}{n^4}=\frac{7\pi^4}{720}
        \end{equation}
        を用いた。
        これらは,Riemann zeta関数$\fun{\zeta}{z}$を用いて一般に,
        \begin{equation}
            \sum_{n=1}^\infty \frac{\pare{-1}^{n-1}}{n^z}=\pare{1-2^{-z}}\fun{\zeta}{z}
        \end{equation}
        と書けて,命題\ref{thm:RiemannZeta-2n}から
        \begin{equation}
            \fun{\zeta}{2}=\frac{\pi^2}{6},\
            \fun{\zeta}{4}=\frac{\pi^4}{90}
        \end{equation}
        であることを用いると導かれる。

        これより,
        \begin{equation}
            \sum_{n=1}^\infty \dfrac{\pare{-1}^{n-1}}{n^3}\fun{\cosech}{\pi n}=\dfrac{\pi^3}{360}
        \end{equation}
        を得る。
    \end{proof}
\end{cor}

\subsection{正弦関数の無限積表示}
\begin{thm}[正弦関数の無限積展開]\label{thm:Poisson-sin}
    \begin{equation}
        \fun{\sin}{x}= x\prod_{n=1}^\infty \pare{1-\dfrac{x^2}{n^2\pi^2}}
    \end{equation}
    \begin{proof}
        $\coth$の展開公式\ref{eg:Poisson-coth}で,$\fun{\coth}{ix}=-i\fun{\cot}{x}$を用いることで,
        \begin{equation}
            \pi\fun{\cot}{\pi x}=\dfrac{1}{x}+\sum_{n=1}^\infty\dfrac{2x}{x^2-n^2}
        \end{equation}
        を得る。
        両辺微分して符号を入れ替えると,
        \begin{equation}
            \dfrac{\pi^2}{\fun{\sin^2}{\pi x}}
            =\dfrac{1}{x^2}+\sum_{n=1}^\infty \pare{
                \dfrac{1}{\pare{x-n}^2}+\dfrac{1}{\pare{x+n}^2}
            }
        \end{equation}
        を得る。
        ここで,
        \begin{equation}
            \fun{f}{x}\coloneqq \pi x \prod_{n=1}^\infty \pare{1-\dfrac{x^2}{n^2}}
        \end{equation}
        と定めて両辺の対数微分を取ると,
        \begin{equation}
            \dfrac{\fun{f'}{x}}{\fun{f}{x}}=\dfrac{1}{x}+\sum_{n=1}^\infty\dfrac{2x}{x^2-n^2}=\pi\fun{\cot}{\pi x}
        \end{equation}
        が分かる。
        両辺積分すると,
        \begin{equation}
            \fun{f}{x}=\pi x \prod_{n=1}^\infty \pare{1-\dfrac{x^2}{n^2}}= \fun{\sin}{\pi x}
        \end{equation}
        が従う\footnote{積分定数を決める必要があるが,例えば$x=\frac{1}{2}$における値としてWallisの公式:$\frac{2}{\pi}=\prod_{n=1}^\infty \pare{1-\frac{1}{4n^2}}$を用いればよい。}。
        よって,
        \begin{equation}
            \fun{\sin}{x}=x\prod_{n=1}^\infty\pare{1-\dfrac{x^2}{n^2\pi^2}}
        \end{equation}
        である。
    \end{proof}
\end{thm}



\subsection{Theta変換公式}
\begin{thm}[Theta変換公式]
    $\alpha$, $t$を正の実数とするとき,
    \begin{equation}
        \sum_{n=-\infty}^\infty \fun{\exp}{-\pi t\pare{n+\alpha}^2}
        =\dfrac{1}{\sqrt{t}}\sum_{n=-\infty}^\infty \fun{\exp}{2\pi i n\alpha -\dfrac{n^2\pi}{t}}
    \end{equation}    
    が成立する。

    \begin{proof}
        $\fun{f}{x}\coloneqq \fun{\exp}{-\pi t \spare{x+\alpha}^2 }$に対してPoisson和公式を適用する。
        $f$のFourier変換:
        \begin{equation}
            \fun{\tilde{f}}{\xi}=\int_{-\infty}^\infty dx\ e^{-\pi t\pare{x+\alpha}^2}e^{2\pi i \xi x}
        \end{equation}
        を求める。
        被積分関数の指数を平方完成すると,
        \begin{align}
            -\pi t \pare{x+\alpha}^2-2\pi i \xi x
            &= -\pi t \bra{
                x^2+2\pare{\alpha+\dfrac{i\xi}{t}}x+\alpha^2
            }
            \\
            &=-\pi t \pare{x+\alpha+\dfrac{i\xi}{t}}-\pi\pare{\dfrac{\xi^2}{t}-2i \alpha \xi}
        \end{align}
        となるので,Gauss積分(命題\ref{prop:gamma-Gauss積分})に帰着させて,
        \begin{align}
            \fun{\tilde{f}}{\xi}
            &=\fun{\exp}{2\pi i \alpha \xi -\dfrac{\pi \xi^2}{t}}\int_{-\infty}^\infty dx\ \fun{\exp}{-\pi t\pare{x+\alpha+\dfrac{i\xi}{t}}^2}
            \\
            &=\fun{\exp}{2\pi i \alpha \xi -\dfrac{\pi \xi^2}{t}}\cdot \sqrt{\pi}\cdot\dfrac{1}{\sqrt{\pi t}}
            \\
            &=\dfrac{1}{\sqrt{t}}\fun{\exp}{2\pi i \alpha \xi-\dfrac{\pi\xi^2}{t}}
        \end{align}
        となる。
        Poisson和公式から,
        \begin{equation}
            \sum_{n=-\infty}^\infty \fun{\exp}{-\pi t\pare{n+\alpha}^2}
        =\dfrac{1}{\sqrt{t}}\sum_{n=-\infty}^\infty \fun{\exp}{2\pi i n\alpha -\dfrac{n^2\pi}{t}}
        \end{equation}
        である。
    \end{proof}
\end{thm}

\begin{cor}[Theta関数]\label{cor:Poisson-Theta変換公式}
    Theta関数$\fun{\vartheta}{x}$を,
    \begin{equation}
        \fun{\vartheta}{x}\coloneqq \sum_{n=1}^\infty e^{-\pi n^2 x} \label{eq:Poisson-Theta関数}
    \end{equation}
    で定めると,
    \begin{equation}
        1+2\fun{\vartheta}{\dfrac{1}{x}}=\sqrt{x}\pare{1+2\fun{\vartheta}{x}}
    \end{equation}
    が成立する。
    \begin{proof}
        Theta変換公式をTheta関数で書き直せば得られる。
    \end{proof}
\end{cor}

Theta関数は楕円関数と関係があり,数論でも重要な役割を果たす。
詳しくは\cite[小山]{koyama}を参照。
\end{document}