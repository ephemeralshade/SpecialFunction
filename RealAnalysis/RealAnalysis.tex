\documentclass[b5paper,draft]{ltjsbook}
%\documentclass[b5paper]{ltjsbook}
\usepackage{../Preamble/preamble}
\begin{document}
\ifdraft{\tableofcontents}{}
\section{実線型空間上のノルム}\seclabel{実線型空間上のノルム}
ノルムに関する以下の定義や命題は\cite{nomura}に拠る。

\begin{defi}[ノルム]\defilabel{実ノルム-定義}
    $V$を$\R$上の線型空間とする。
    $\abs{\dummy}$を実数の絶対値とする。
    関数$\norm{\dummy}\colon V\to \R$が以下の条件を満たすとき,$\norm{x}$を$x$のノルムという:
    \begin{enumerate}[label=(\roman*)]
        \item (正値性) $\forall x\in V,\ \norm{x}\geq 0$
        \item (一意性) $\forall x\in V,\ \bsqu{\norm{x}=0\ \Longleftrightarrow\ x=0}$
        \item (同次性) $\forall k\in \R,\ \forall x\in V,\ \norm{kx}=\abs{k}\norm{x}$
        \item (三角不等式) $\forall x, \forall y\in V,\ \norm{x+y}\leq \norm{x}+\norm{y}$
    \end{enumerate}
\end{defi}

ノルムの定義された線型空間をノルム空間という。


\begin{prop}[三角不等式]\proplabel{実ノルム-三角不等式}
    ノルムは次の性質も満たす:
    \begin{equation}
        \forall x, \forall y\in V,\ \abs{\norm{x}-\norm{y}}\leq \norm{x-y}.
    \end{equation}
    \begin{proof}
        まず,$\norm{x}=\norm{\pare{x-y}+y}\leq \norm{x-y}+\norm{y}$
        から$\norm{x}-\norm{y}\leq \norm{x-y}$
        が従う。
        同様に
        \begin{equation}
            \norm{y}=\norm{\pare{y-x}+x}\leq \norm{\pare{-1}\pare{x-y}}+\norm{x}=\abs{-1}\norm{x-y}+\norm{x}=\norm{x-y}+\norm{x}
        \end{equation}
        なので,$\norm{y}-\norm{x}\leq \norm{x-y}$である。
        また,絶対値の特徴付け$\abs{a}= \max\bra{a,-a}$
        を用いると,$ \max\bra{\norm{x}-\norm{y},\norm{y}-\norm{x}} = \abs{\norm{x}-\norm{y}}\leq \norm{x-y}$である。
        これらにより,$\abs{\norm{x}-\norm{y}}\leq \norm{x-y}$が成立する。
    \end{proof}
\end{prop}

\begin{cor}\corlabel{実ノルム-三角不等式}
    \propref{実ノルム-三角不等式}において,$y\mapsto -y$として,元の三角不等式と併せれば,
    \begin{equation}
        \abs{\norm{x}-\norm{y}}\leq \norm{x+y}\leq \norm{x}+\norm{y}
    \end{equation}
    が従う。
\end{cor}

\begin{prop}[ノルムの連続性]\proplabel{実ノルム-連続性}
    ノルムは連続関数である。
    \begin{proof}
        示すべきことは以下の通り(\defiref{Lipschitz-関数の連続性}も参照):
        \begin{equation}
            \forall x,\forall y\in V,\ \forall \varepsilon\in\Rposi,\
                \exists \delta\in \Rposi;\ \bsqu{
                    \norm{x-y}<\delta\, \Longrightarrow\, \abs{\norm{x}-\norm{y}}<\varepsilon
                }.
        \end{equation}

        実際,$\delta\coloneqq \varepsilon$と取れば,\propref{実ノルム-三角不等式}により,$\abs{\norm{x}-\norm{y}}\leq \norm{x-y}<\delta =\varepsilon$であるから,ノルムは連続関数である。
    \end{proof}
\end{prop}

\begin{defi}[ノルムの同値性]\defilabel{実ノルム-ノルムの同値性}
    以下が成立するとき,$V$上の2つのノルム$\norm{\dummy}$, $\norm{\dummy}_1$が同値であるという:
    \begin{equation}
        \exists m, \exists M\in\R;\ \bsqu{0<m\leq M\ \land \ \pare{
            \forall x\in V,\ m\norm{x}_1 \leq \norm{x} \leq M\norm{x}_1
        }}.
    \end{equation}
\end{defi}

\begin{prop}\proplabel{実ノルム-同値関係}
    ノルムの同値性は同値関係である。
    \begin{proof}
        ノルム$\norm{\dummy}$, $\norm{\dummy}_1$が同値であることを$\norm{\dummy}\sim\norm{\dummy}_1$と記すことにする。
        すなわち,
        \begin{equation}
            \norm{\dummy}\sim\norm{\dummy}_1\ \defarrow\ \exists m, \exists M\in\R;\ \bsqu{0<m\leq M\ \land \ \pare{
                \forall x\in V,\ m\norm{x}_1 \leq \norm{x} \leq M\norm{x}_1
            }}.
        \end{equation}
        このとき関係$\sim$が反射律,対称律,推移律を満たすことを示す。
        \begin{itemize}
            \item (反射律)
            任意の$\norm{\dummy}$に対し,$\norm{\dummy}\sim\norm{\dummy}$が成立することを示す。
            実際,$m=M=1$とすれば,
            \begin{equation}
                0<m=1\leq M=1\ \land\ \pare{\forall x\in V,\ 1\cdot \norm{x}\leq \norm{x}\leq 1\cdot \norm{x}}
            \end{equation}
            であるから成立。

            \item (対称律)
            $\norm{\dummy}\sim\norm{\dummy}_1$,すなわち
            \begin{equation}
                \exists m, \exists M\in\R;\ \bsqu{0<m\leq M\ \land \ \pare{
                \forall x\in V,\ m\norm{x}_1 \leq \norm{x} \leq M\norm{x}_1
                }}
            \end{equation}
            のとき,$\norm{\dummy}_1\sim\norm{\dummy}$が成立することを示す。
            後半の不等式を書き直すと,$\frac{1}{M}\norm{x}\leq \norm{x}_1\leq \frac{1}{m}\norm{x}$であり,$0<m\leq M$であるから$0<\frac{1}{M}\leq \frac{1}{m}$であり,$m' \coloneqq \frac{1}{M}$, $M' \coloneqq  \frac{1}{m}$と取れば確かに$\norm{\dummy}_1\sim\norm{\dummy}$の成立が判る。

            \item (推移律)
            3つのノルム$\norm{\dummy}$, $\norm{\dummy}_1$, $\norm{\dummy}_2$に対し,$\norm{\dummy}\sim \norm{\dummy}_1$と$\norm{\dummy}_1\sim \norm{\dummy}_2$のとき,$\norm{\dummy}\sim \norm{\dummy}_2$が成立することを示す。
            仮定から,
            \begin{align}
                &\exists m_1, \exists M_2\in\R;\ \bsqu{0<m_1\leq M_2\ \land \ \pare{
                \forall x\in V,\ m_1\norm{x}_1 \leq \norm{x} \leq M_1\norm{x}_1
                }},
                \\
                &\exists m_2, \exists M_2\in\R;\ \bsqu{0<m_2\leq M_2\ \land \ \pare{
                \forall x\in V,\ m_2\norm{x}_2 \leq \norm{x}_1 \leq M_2\norm{x}_2
                }}
            \end{align}
            の両方が成立しているから,$\norm{x}\leq M_1\norm{x}_1\leq M_1M_2\norm{x}_2$及び,$\norm{x}\geq m_1\norm{x}_1\geq m_1m_2 \norm{x}_2$より,$m_1m_2\norm{x}_2\leq \norm{x}\leq M_1M_2\norm{x}_2$が従う。
            これより,$m\coloneqq m_1m_2$, $M\coloneqq M_1M_2$と取れば$0<m_1m_2\leq M_1M_2$も成立するので$\norm{\dummy}\sim \norm{\dummy}_2$である。
        \end{itemize}

        以上により,ノルムの同値性は同値関係である。
    \end{proof}
\end{prop}

\begin{nota}[上添字]
    $\R^n$のベクトル$x$の成分の添字は上に付けて表すことにする。
    すなわち,$x=(x^0,x^1,\dots,x^{n-1})$のように記す。
    成分の冪を表すわけではないことに注意。
    ベクトルの添字を$x_0=(x^0_0,x^1_0,\dots,x^{n-1}_0)$, $x_1=(x^0_1,x^1_1,\dots,x^{n-1}_1),\dots$のように下に付けて記した場合,複数あるベクトルのいずれかの番号を意味する。
\end{nota}

\begin{thm}\thmlabel{実ノルム-同値性}
    線型空間$V$が有限次元であれば,$V$上の任意の2つのノルムは同値である。
    \begin{proof}
        $V$を有限次元線型空間とし,$n\coloneqq \dim{V}$とする。
        $V$の基底として,$\seq{\ele_k}{0\leq k\leq n-1}$を取り固定する。
        $V$の元$x$を,$x =\sum_{k=0}^{n-1} x^k \ele_k$と表したときの成分$\seq{x^k}{0\leq k\leq n-1}$を用いて,
        \begin{equation}
            \norm{\dummy}\colon V\to \R;\ x\mapsto \norm{x}\coloneqq \max_{0\leq k\leq n-1}\bra{\abs{x^k}}
            \equlabel{実ノルム-max}
        \end{equation}
        と定めると$\norm{\dummy}$はノルムになる。
        以下$\norm{\dummy}$がノルムであることを確かめる:
        \begin{enumerate}[label=(\roman*)]
            \item (正値性)
            どの$k$についても$0\leq \abs{x^k}$であり,$\displaystyle\norm{x}= \max_k\bra{\abs{x^k}}\geq 0$であるから成立。

            \item (一意性)
            $x=0$のとき,どの$k$についても$x^k=0$であるから$\displaystyle\norm{x}= \max_k\bra{\abs{x^k}}=0$である。
            また,$\norm{x}=0$のとき,任意の$k$に対し,絶対値の非負性から$0\leq \abs{x^k}$であって,$\abs{x^k}\leq \norm{x}=0$であるから$\abs{x^k}=0$である。
            これより$x=0$となる。

            \item (同次性)
            絶対値の非負性と同次性から,$\displaystyle\norm{cx}=\max_k \bra{\abs{cx}}=\abs{c}\cdot \max_k\bra{\abs{x^k}}=\abs{c}\norm{x}$が従う。

            \item (三角不等式)
            絶対値の三角不等式から,各$k$に対して$\abs{x^k+y^k}\leq \abs{x^k}+\abs{y^k}$であるから最大値に関してもこの不等号が成り立つので$\norm{x+y}\leq \norm{x}+\norm{y}$が従う。
        \end{enumerate}
        以上により\equref{実ノルム-max}で定められた$\norm{\dummy}$はノルムである。

        次に,$V$上の勝手なノルム$\norm{\dummy}_1$を取ってきたときに,\equref{実ノルム-max}で定義した$\norm{\dummy}_1$と$\norm{\dummy}$が同値になってしまうことを示す。
        $V$のコンパクト集合$S$を$S\coloneqq \sset{y\in V}{\norm{y}=1}$
        によって定め,関数$f\colon S\to \R$を$\fun{f}{y}\coloneqq \norm{y}_1$と定める。
        $f$の連続性(\propref{実ノルム-連続性})と$S$がコンパクト集合であることから,$f$の値域には最小元$m$と最大元$M$が存在する。
        $S$上で$y\ne {0}$であり,ノルムの正値性から$0<m\leq M$である。
        特に,$\norm{y}=1$ならば$y\in S$であり,$m$, $M$の定義から$m\leq \fun{f}{y}\leq M$及び$\fun{f}{y}=\norm{y}_1$なので$m\leq \norm{x}_1\leq M$が従う。
        ここで,$V$上の一般の$x\ne 0$に対して$y\coloneqq x / \norm{x}$とすると$\norm{y}=\norm{x/\norm{x}}=\norm{x}/\norm{x}=1$より$y\in S$であるから$\fun{f}{y}=\norm{x/\norm{x}}_1=\norm{x}_1/\norm{x}$であって,$m\leq \norm{x}_1/\norm{x}\leq M$より$m\norm{x}\leq \norm{x}_1\leq M\norm{x}$である。
        $x=0$についても$\norm{0}=\norm{0}_1=0$であって,$m \norm{0}_1\leq \norm{0}\leq M\norm{0}_1$は成立するので,$\norm{\dummy}_1$と$\norm{\dummy}$は同値である。

        有限次元線型空間$V$上で与えられた任意の2つのノルム$\norm{\dummy}_1$, $\norm{\dummy}_2$はそれぞれ\equref{実ノルム-max}で定義された$\norm{\dummy}$と同値であり,ノルムの同値は同値関係(\propref{実ノルム-同値関係})なので,$\norm{\dummy}_1$と$\norm{\dummy}_2$も同値である。
    \end{proof}
\end{thm}


\subsection{実線型空間上のノルム}\subseclabel{実ノルム-実線型空間上のノルム}
上ではノルムに関して,一般的な形でやや抽象的に述べた。
以降では\cite{takano}に基づいて,常微分方程式を実際に扱う際に有用なノルムを具体的に扱う。
ここでは有限次元実線型空間$\R^n$上のノルムの例を挙げる。
以下では,$x=\pare{x^0,x^1,\dots,x^{n-1}}$の各成分を実数とし,$\abs{\dummy}$を実数の絶対値とする。

\begin{eg}[Euclidノルム]\eglabel{実ノルム-Euclidノルム}
    原点からの標準的な距離:
    \begin{equation}
        \norm{x}_2\coloneqq \pare{\sum_{k=0}^{n-1} \abs{x^k}^2}^{\frac{1}{2}}
    \end{equation}
    はノルムである。
\end{eg}

\begin{eg}[一様ノルム]\eglabel{実ノルム-一様}
    \thmref{実ノルム-同値性}の証明において\equref{実ノルム-max}で定義された
    $\norm{\dummy}$はノルムである。
    後述するように,$p$ノルムにおいて$p\to\infty$の極限で再現されることから,これを$\norm{\dummy}_\infty$と書いて$\infty$ノルムとも呼ぶ。
\end{eg}

\begin{defi}[$p$ノルム]\defilabel{実ノルム-pノルム}
    $p$を$1$以上の実数とする。
    このとき,$x$の$p$ノルム$\norm{x}_p$を,
    \begin{equation}
        \norm{x}_p \coloneqq \pare{\sum_{k=0}^{n-1} \abs{x^k}^p}^{\frac{1}{p}}
    \end{equation}
    で定める。
\end{defi}

\begin{prob}\problabel{実ノルム-pノルムがノルムであることの証明}
    $p$ノルムがノルムの条件(\defiref{実ノルム-定義})を満たしていることを確かめよ。
\end{prob}

$1$ノルムのことを絶対値ノルムと呼び,絶対値ノルムによって定まる距離をManhattan距離と呼ぶ。
$2$ノルムはEuclidノルムに一致する。

\begin{prop}\proplabel{実ノルム-pノルムが一様ノルムに収束する}
    $p\to\infty$の極限で$p$ノルム$\norm{\dummy}_p$は\egref{実ノルム-一様}の一様ノルム$\norm{\dummy}_\infty$に一致する。
    \begin{proof}
        $x^M$は$x$の成分の$x^0,x^1,\dots,x^{n-1}$のうち絶対値が最大のものであるとする。
        このとき,$\abs{x^M}=\norm{x}_\infty$と書ける。
        定義より,
        \begin{equation}
            \norm{x}_p=\pare{\sum_{k=0}^{n-1}\abs{x^k}^p}^{\frac{1}{p}},\quad
            \norm{x}_\infty=\max_{0\leq k\leq n-1}\bra{\abs{x^k}}=\abs{x^M}
        \end{equation}
        であるから,
        示すべきことは,
        \begin{equation}
            \pare{\sum_{k=0}^{n-1}\abs{x^k}^p}^{\frac{1}{p}}\to \abs{x^M}\quad\pare{p\to\infty}
            \equlabel{実ノルム-pノルムの極限}
        \end{equation}
        である。
        まず,$1<p<\infty$のとき,$\norm{x}_\infty \leq \norm{x}_p$であることを確かめる。
        これは,
        \begin{equation}
            0\leq \abs{x^M}^p\leq \sum_{k=0}^{n-1}\abs{x^k}^p
        \end{equation}
        であることと,$0<p$のとき,$t\in \mleft[0,\infty\mright)$において$\fun{f}{t}=t^{\frac{1}{p}}$が狭義単調増加であること,$0$以上$(n-1)$以下の任意の整数$k$について$\abs{x^k}\geq 0$であることから,
        \begin{equation}
            \abs{x^M}=\pare{\abs{x^M}^p}^{\frac{1}{p}}\leq \pare{\sum_{k=0}^{n-1}\abs{x^k}^p}^{\frac{1}{p}}
        \end{equation}
        より成立。
        また,$0$以上$(n-1)$以下の任意の整数$k$について$0\leq \abs{x^k}\leq \abs{x^M}$であり,
        \begin{equation}
            0\leq \sum_{k=0}^{n-1} \abs{x^k}^p\leq \sum_{k=0}^{n-1}\abs{x^M}^p=n\abs{x^M}^p
        \end{equation}
        が成立する。
        更に,$n$は固定された正整数であるから
        \begin{equation}
            n^{\frac{1}{p}}\to n^0=1\quad\pare{p\to\infty}
        \end{equation}
        である。

        これらのことから,
        \begin{equation}
            0\leq \pare{\sum_{k=0}^{n-1}\abs{x^k}^p}^{\frac{1}{p}}-\abs{x^M}
            \leq \pare{n\abs{x^M}^p}^{\frac{1}{p}}-\abs{x^M}
            \leq \pare{n^{\frac{1}{p}}-1}\abs{x^M}
            \to 0\quad\pare{p\to\infty}
        \end{equation}
        が従い,はさみうちの原理から,
        \begin{equation}
            \pare{\sum_{k=0}^{n-1}\abs{x^k}^p}^{\frac{1}{p}}\to \abs{x^M}\quad\pare{p\to\infty}
        \end{equation}
        であり,これは\equref{実ノルム-pノルムの極限}の成立を意味している。
    \end{proof}
\end{prop}

以降,有限次元線型空間のノルムは\egref{実ノルム-一様}の一様ノルムであるとする。

\begin{nota}\notalabel{実ノルム-n-1以下の整数}
    以下では簡単のために,$n-1$以下の非負整数の集合を,
    \begin{equation}
        \setN\coloneqq \set{j}{j\in\N\ \land\  0\leq j\leq n-1}
    \end{equation}
    と書く。
\end{nota}


\begin{prop}\proplabel{実ノルム-積分の不等式}
    $I$を$\R$上の閉区間であるとする。
    $f\colon I\to \R^n$を連続なベクトル値関数とするとき,
    \begin{equation}
        \norm{\int_I dt\, \fun{f}{t}}\leq \int_I dt\, \norm{\fun{f}{t}}
    \end{equation}
    が成立する。
    \begin{proof}
        $I$上可積分な$g\colon I\to \R$と絶対値に関する以下の不等式:
        \begin{equation}
            \abs{\int_I dt\, \fun{g}{t}}\leq \int_I dt\, \abs{\fun{g}{t}}
            \equlabel{実ノルム-積分絶対値不等式}
        \end{equation}
        を用いる。
        \begin{equation}
            \forall j\in \setN,\, \forall t\in I,\ 
            \fun{f^j}{t}\leq \abs{\fun{f^j}{t}}\leq \norm{\fun{f}{t}}
        \end{equation}
        であるから,
        \begin{equation}
            \forall j\in \setN,\ 
            \int_I dt\, \fun{f^j}{t}\leq \int_I dt\, \abs{\fun{f^j}{t}}
            \leq \int_I dt\, \norm{\fun{f}{t}}
        \end{equation}
        である。
        ここで,\equref{実ノルム-積分絶対値不等式}の不等式から,
        \begin{equation}
            \abs{\int_I dt\, \fun{f^j}{t}}\leq \int_I dt\, \abs{\fun{f^j}{t}}\leq \int_I dt\, \norm{\fun{f}{t}}
        \end{equation}
        であるから,
        \begin{equation}
            \norm{\int_I dt \, \fun{f}{t}}
            =\max_{0\leq j\leq n-1}\bra{\abs{\int_I dt\, \fun{f^j}{t}}}\leq \int_I dt\, \norm{\fun{f}{t}}
        \end{equation}
        が成立する。
    \end{proof}
\end{prop}

\if0
\begin{lem}
    $S_1$, $S_2$を適当な集合とし,写像$f\colon S_1\times S_2\to \R$を与え,$T_1$, $T_2$をそれぞれ$S_1$, $S_2$の部分集合とする。
    このとき,
    \begin{equation}
        \sup_{a\in T_1}\bra{\max_{b\in T_2}\bra{\fun{f}{a,b}}}
        =\max_{b\in T_2}\bra{\sup_{x\in a_1}\bra{\fun{f}{a,b}}}
    \end{equation}
    が成立する。
    \begin{proof}
        \begin{prooftree}
            {
                close with={\times},
                to prove={
                    \pare{\pare{\forall a} \pare{p(a) \to \pare{\pare{\forall b} \pare{ q(b) \to r(a,b)}}}}
                    \vdash
                    \pare{\pare{\forall b} \pare{q(b) \to \pare{\pare{\forall a} \pare{ p(a) \to r(a,b)}}}},
                }
            }
            [
                \pare{\forall a} \pare{p(a) \to \pare{\pare{\forall b} \pare{ q(b) \to r(a,b)}}}, just={Assumption}, name={1}
                    [\lnot\pare{\pare{\forall b} \pare{q(b) \to \pare{\pare{\forall a} \pare{ p(a) \to r(a,b)}}}}, just={$\lnot$ Conclusion}
                    [\lnot\pare{q(b')\to\pare{\pare{\forall a}\pare{p(a)\to r(a,b')}}}, just={$\lnot\pare{\forall b}$ {}:!u}
                        [q(b'), just={$\lnot \to$ {}:!u}, name={4}
                            [\lnot\pare{\pare{\forall a}\pare{p(a)\to r(a,b')}}, just={$\lnot \to$ {}:!uu}
                                [\lnot\pare{p(a')\to r(a',b')}, just={$\lnot\pare{\forall a}$ {}:!u}
                                    [p(a'), just={$\lnot \to$ {}:!u}, name={7}
                                        [{\lnot r(a',b')}, just={$\lnot\to$ {}:!uu}, name={8}
                                            [p(a') \to \pare{\pare{\forall b}\pare{q(b)\to r(a',b)}}, just={$\pare{\forall a}$ {}:1} 
                                                [\lnot p(a'), just={$\to$ {}:!u}, close={:7,!c}]
                                                [\pare{\forall b}\pare{q(b)\to r(a',b)}
                                                [{q(b')\to r(a',b')}, just={$\pare{\forall b}$ {}:!u}
                                                    [\lnot q(b'), , just={$\to$:!u},close={:4,!c}]
                                                    [{r(a',b')}, close={:8,!c}]
                                                ]
                                                ]
            ]]]]]]]]]
        \end{prooftree}
    \end{proof}
\end{lem}
\fi

\begin{defi}[作用素ノルム]\defilabel{実ノルム-作用素ノルム}
    線型作用素$A\colon \R^n\to\R^n$に対して,$\norm{A}$を,
    \begin{equation}
        \norm{A}\coloneqq \sup_{\norm{x}=1}\bra{\norm{Ax}}
    \end{equation}
    で定める。
\end{defi}

以降,$A$を線型作用素とし,その$\pare{j,k}$成分を$a^j_k$と書く。

\begin{prop}\proplabel{実ノルム-作用素ノルム具体形}
    \begin{equation}
        \norm{A} = \max_{0\leq j\leq n-1} \bra{\sum_{k=0}^{n-1} \abs{a^j_k}}
    \end{equation}
    \begin{proof}
        $\norm{x}=1$のとき,$x$の各成分の絶対値は$1$以下である。
        このとき,$Ax$の第$j$成分$\pare{Ax}_j$に関して,
        \begin{equation}
            \abs{\pare{Ax}_j}=\abs{\sum_{k=0}^{n-1}a^j_k x^k}
            \leq \sum_{k=0}^{n-1}\abs{a^j_k}\abs{x^k}\leq \sum_{k=0}^{n-1}\abs{a^j_k}
        \end{equation}
        となり,1つ目の等号成立は,
        \begin{equation}
            \bsqu{\forall k\in\setN,\ a^j_kx^k\geq 0}
            \ \lor\ 
            \bsqu{\forall k\in\setN,\ a^j_kx^k\leq 0}
        \end{equation}
        のとき,すなわち$a^j_kx^k$が全て同符号のときである。
        2つ目の等号成立は,$x$の各成分の絶対値が$1$に等しい場合であり,この2つの条件は同時に満たすことができる。
        実際,各$k$について,$a^j_k\geq 0$なら$x^k=1$とし,$a^j_k<0$なら$x^k=-1$と定めれば,いずれの等号も成立する。
        これより,
        \begin{equation}
            \norm{A} = \max_{0\leq j\leq n-1} \bra{\sum_{k=0}^{n-1} \abs{a^j_k}}
        \end{equation}
        である。
    \end{proof}
\end{prop}

\begin{prop}\proplabel{実ノルム-作用素ノルム不等式1}
    \begin{equation}
        \forall x\in \R^n,\ \norm{Ax}\leq \norm{A}\norm{x}
    \end{equation}
    \begin{proof}
        各$k$について$\abs{x^k}\leq \norm{x}$であるから,
        $Ax$の第$j$成分について,
        \begin{equation}
            \abs{\pare{Ax}_j}=\abs{\sum_{k=0}^{n-1}a^j_kx^k}
            \leq \sum_{k=0}^{n-1} \abs{a^j_k}\abs{x^k}
            \leq \pare{\sum_{k=0}^{n-1}\abs{a^j_k}}\cdot \norm{x}
        \end{equation}
        が成立して,
        \begin{equation}
            \forall x\in \R^n,\ \norm{Ax}\leq \norm{A}\norm{x}
        \end{equation}
        が従う。
    \end{proof}
\end{prop}

\begin{prop}\proplabel{実ノルム-作用素ノルム三角不等式}
    $A$, $B$を$\R^n$から$\R^n$への任意の線型作用素とする。
    このとき,
    \begin{equation}
        \norm{A+B}\leq \norm{A}+\norm{B},\ \norm{AB}\leq \norm{A}\norm{B}
    \end{equation}
    が成立する。
    \begin{proof}
        $A$, $B$の$\pare{j,k}$成分をそれぞれ$a^j_k$, $b^j_k$とする。
        まず,
        \begin{equation}
            \sum_{k=0}^{n-1}\abs{a^j_k+b^j_k}
            \leq \sum_{k=0}^{n-1}\abs{a^j_k}
            +\sum_{k=0}^{n-1}\abs{b^j_k}
        \end{equation}
        であるから,
        \begin{equation}
            \max_{0\leq j\leq n-1}\bra{\sum_{k=0}^{n-1}\abs{a^j_k+b^j_k}}
            \leq \max_{0\leq j\leq n-1}\bra{\sum_{k=0}^{n-1}\abs{a^j_k}}
            +\max_{0\leq j\leq n-1}\bra{\sum_{k=0}^{n-1}\abs{b^j_k}}
        \end{equation}
        であり,$\norm{A+B}\leq \norm{A}+\norm{B}$である。
        
        次に,$m$は,
        \begin{equation}
            \forall \ell\in \setN,\
            \sum_{k=0}^{n-1}\abs{b^{\ell}_k}\leq
            \sum_{k=0}^{n-1}\abs{b^m_k}
        \end{equation}
        を満たす,すなわち$\norm{B}$を与える行の添字とする。
        このとき,
        \begin{equation}
            \sum_{k=0}^{n-1}\abs{\sum_{\ell=0}^{n-1}a^j_\ell b^\ell_k}
            \leq \sum_{k=0}^{n-1}\pare{\sum_{\ell=0}^{n-1}\abs{a^j_\ell }\abs{b^\ell_k}}
            =\sum_{\ell=0}^{n-1}\pare{\abs{a^j_\ell }\sum_{k=0}^{n-1}\abs{b^\ell_k}}
            \leq \pare{\sum_{\ell=0}^{n-1}\abs{a^j_\ell }}
            \pare{\sum_{k=0}^{n-1}\abs{b^m_k}}
        \end{equation}
        より,
        \begin{equation}
            \max_{0\leq j\leq n-1}\bra{\sum_{k=0}^{n-1}\abs{\sum_{\ell=0}^{n-1}a^j_\ell b^\ell_k}}
            \leq \max_{0\leq j\leq n-1}\bra{\sum_{\ell=0}^{n-1}\abs{a^j_\ell }}\cdot \max_{0\leq j\leq n-1}\bra{\sum_{k=0}^{n-1}\abs{b^j_k}}
        \end{equation}
        が従うので,
        $\norm{AB}\leq \norm{A}\norm{B}$である。
    \end{proof}
\end{prop}

\begin{prop}\proplabel{実ノルム-作用素ノルムはノルム}
    \defiref{実ノルム-作用素ノルム}で定義した$\norm{A}$はノルムである。
    \begin{proof}
        及び,\propref{実ノルム-作用素ノルム三角不等式}
        \defiref{実ノルム-定義}の4条件を順に確かめる。
        \begin{enumerate}[label=(\roman*)]
            \item (正値性)
            \propref{実ノルム-作用素ノルム具体形}から,各項にある絶対値は非負なので,$\norm{A} \geq 0$である。

            \item (一意性)
            \propref{実ノルム-作用素ノルム具体形}から,$\norm{A}=0$だとすれば,絶対値の非負性から,どの$j$に関しても$\sum_{k=0}^{n-1}\abs{a^j_k}=0$であり,これが成立するためには全ての$\pare{j,k}$に関して$A_{jk}=0$でなければならず,このとき$A=O$(零行列)である。

            \item (同次性)
            \propref{実ノルム-作用素ノルム具体形}から,
            \begin{equation}
                \norm{cA}=\max_{0\leq j\leq n-1} \bra{\sum_{k=0}^{n-1} \abs{c}\abs{a^j_k}}=\abs{c}\cdot \max_{0\leq j\leq n-1} \bra{\sum_{k=0}^{n-1} \abs{a^j_k}}=\abs{c}\norm{A}.
            \end{equation}

            \item (三角不等式)
            \propref{実ノルム-作用素ノルム三角不等式}から直ちに従う。
        \end{enumerate}
    \end{proof}
\end{prop}

\begin{rem}
    上で述べたことは複素線型空間上のノルムでもほとんど同様に成り立つ。
    証明も,$\abs{\dummy}$を複素数の絶対値に読み替えれば成立するが,\propref{実ノルム-作用素ノルム具体形}の証明中の等号成立条件だけ注意が必要である。
\end{rem}


\begin{defi}[$L^p$ノルム]\defilabel{実ノルム-Lpノルム}
    $I$を$\R$上の区間とし,$f\colon I\to \CC$は$I$上可積分とし,$1\leq p <\infty$する。
    このとき,$f$のノルム$\norm{f}_p$を,
    \begin{equation}
        \norm{f}_p \coloneqq \pare{\int_I dx\, \abs{\fun{f}{x}}^p}^{\frac{1}{p}}
    \end{equation}
    と定める。
\end{defi}

\begin{supple}
    \defiref{実ノルム-Lpノルム}は多変数ベクトル値関数に拡張することができる。
    詳細は\cite{nomura}など。
\end{supple}

\begin{defi}[$L^\infty$ノルム]\defilabel{実ノルム-Linftyノルム}
    $I$を$\R$上の区間とし,$f\colon I\to \CC$は$I$上可積分とし,$1\leq p <\infty$する。
    このとき,$f$のノルム$\norm{f}_\infty$を,
    \begin{equation}
        \norm{f}_\infty \coloneqq \sup_{x \in I}\bra{ \abs{\fun{f}{x}} }
    \end{equation}
    と定める。
\end{defi}

\begin{rem}
    \defiref{実ノルム-pノルム, 実ノルム-Lpノルム}及び\defiref{実ノルム-一様, 実ノルム-Linftyノルム}について,一般には$\norm{f}_p\ne\norm{\fun{f}{x}}$及び$\norm{f}_\infty\ne\norm{\fun{f}{x}}_\infty$であることに注意。
    $\norm{f}$は関数のノルムであるが,$\norm{\fun{f}{x}}$は$f$に$x$を入れた$\fun{f}{x}$というベクトルに対するノルムである。
\end{rem}


\begin{prob}\problabel{実ノルム-その1}
    $\R^n$のベクトル$x$のノルムは\egref{実ノルム-一様}で述べた一様ノルムとし,
    $\R^n$上の線型作用素$A$のノルムは\defiref{実ノルム-作用素ノルム}で与えられた作用素ノルムであるとする。
    $n=3$のとき,
    \begin{equation}
        x=\vctr{-1\\3\\-4},\
        A=\pmat{1 & -1 & 4\\
            2 & -2 & 3\\
            -2 & -3 & 0
            }
    \end{equation}
    とする。
    このとき,$\norm{x}$と$\norm{A}$を計算せよ。
\end{prob}

\begin{prob}\problabel{実ノルム-その2}
    $\R^n$のベクトルに対して一様ノルムではなく,\egref{実ノルム-Euclidノルム}で導入したEuclidノルムを用いた場合,\defiref{実ノルム-作用素ノルム}で定義される$\norm{A}$はどのような性質を持つだろうか。
    例えば,$\R^2$上の一般的な線型作用素の作用素ノルムを,その成分を用いて具体的に表わせ。
    (関連:スペクトルノルム)
\end{prob}

%%% 複素数上でのノルムは次のchapterで

\section{Lipschitz連続}\seclabel{Lipschitz連続}
ここでは関数の連続性,一様連続性,Lipschitz連続性について述べる。
参考文献は\cite{sugiura}である。
$m$, $n$を正整数とする。
$A\subseteq \R^m$とし,$f\colon A\to \R^n$について考える。

\begin{defi}[連続性]\defilabel{Lipschitz-関数の連続性}
    $f$が$A$上連続であるとは,
    \begin{equation}
        \forall \varepsilon\in\Rposi,\, \forall x\in A,\, \exists \delta \in \Rposi;\, \forall y \in A,\ \bsqu{
            \norm{x-y}<\delta \ \Longrightarrow\ \norm{\fun{f}{x}-\fun{f}{y}}<\varepsilon
        }
    \end{equation}
    が成立することである。
\end{defi}

\begin{defi}[一様連続性]
    $f$が$A$上一様連続であるとは,
    \begin{equation}
        \forall \varepsilon\in\Rposi,\, \exists \delta \in \Rposi;\,  \forall x, \forall y \in A,\ \bsqu{
            \norm{x-y}<\delta \ \Longrightarrow\ \norm{\fun{f}{x}-\fun{f}{y}}<\varepsilon
        }
    \end{equation}
    が成立することである。
\end{defi}

\begin{defi}[Lipschitz連続]\defilabel{Lipschitz-Lipschitz連続}
    $f$が$A$上Lipschitz連続であるとは,
    \begin{equation}
        \exists L\in \Rnonnega;\, \forall x, \forall y \in A,\ \bsqu{
            \norm{\fun{f}{x}-\fun{f}{y}}\leq L \norm{x - y}
        }
    \end{equation}
    が成立することである。
    このとき,$L$をLipschitz定数という。
\end{defi}

\begin{rem}
    Lipschitz定数は一意的ではない。
    実際,$L$がLipschitz定数のとき,$c>1$を用いると$\norm{\fun{f}{x}-\fun{f}{y}}\leq L \norm{x - y}\leq cL \norm{x-y}$も成立するので,$(L\neq )cL$もLipschitz定数である。
    上の条件を満たすLipschitz定数全体の下限を改めてLipschitz定数と呼ぶことにすれば,これは一意である。
\end{rem}

\begin{prop}\proplabel{Lipschitz-Lipschitz連続なら一様連続}
    $A$上Lipschitz連続な関数は$A$上一様連続である。
    \begin{proof}
        $f$を$A$上Lipschitz連続であると仮定し,Lipschitz定数の一つを$L$と記す。
        任意の正数$\varepsilon$に対して,$\delta \coloneqq \varepsilon / \pare{L+1}$と置けば,$\delta >0$であり,$\norm{x-y}<\delta = {\varepsilon}/\pare{L+1}$のとき,$L\norm{x-y}<\pare{L+1}\norm{x-y}<\varepsilon$であり,
        \begin{equation}
            \norm{\fun{f}{x} - \fun{f}{y}} \leq L \norm{x - y}<\pare{L+1}\norm{x-y}<\varepsilon
        \end{equation}
        が成立するので一様連続である。
    \end{proof}
\end{prop}

\begin{prop}\proplabel{Lipschitz-一様連続なら連続}
    $A$上一様連続な関数は$A$上連続である。
    \begin{proof}
        一般に,$\fun{P}{a,b}$を$a$, $b$に関する任意の命題としたとき,
        \begin{equation}
            \squ{\exists a;\, \forall b,\, \fun{P}{a,b}} \ \Longrightarrow\ \squ{\forall b,\, \exists a;\, \fun{P}{a,b}}
            \equlabel{Lipschitz-恒真命題 exists forall then forall exists}
        \end{equation}
        は恒真である。
        証明木を描いてみると,
        \figref{Lipschitz-恒真命題 exists forall then forall existsの証明木}に示すように,\equref{Lipschitz-恒真命題 exists forall then forall exists}が恒真であることが判る。
        \ifdraft{}{
        \begin{figure}[H]
            \centering
            \begin{prooftree}
                {
                    close with={\times},
                    to prove={
                        \squ{\exists a;\, \forall b,\, \fun{P}{a,b}}
                        \vdash
                        \squ{\forall b,\, \exists a;\, \fun{P}{a,b}}
                    }
                }
                [
                    {\exists a;\, \forall b,\, \fun{P}{a,b}}, just={Assumption}, name={Assumption}
                    [\lnot\squ{\forall b,\, \exists a;\, \fun{P}{a,b}}, just={$\lnot$ Conclusion}
                    [{\exists b;\, \lnot\squ{\exists a,\, \fun{P}{a,b}}}, just={$\lnot \forall b$ {}:!u}
                    [{\exists b;\, \forall a,\, \lnot\fun{P}{a,b}}, just={$\lnot \exists a$ {}:!u}
                    [{\forall b,\, \fun{P}{a',b}}, just={$\exists a$ {}:Assumption}
                    [{\forall a,\, \lnot \fun{P}{a,b'}}, just={$\exists b$ {}:!uu}
                    [{\fun{P}{a',b'}}, just={$\forall b$ {}:!uu}
                    [{\lnot\fun{P}{a',b'}}, just={$\forall a$ {}:!uu}, close={{}:!u,!c}
                    ]]]]]]]]
            \end{prooftree}
            \caption{\equref{Lipschitz-恒真命題 exists forall then forall exists}の証明木}
            \figlabel{Lipschitz-恒真命題 exists forall then forall existsの証明木}
        \end{figure}}
        一様連続性では$\exists \delta;\, \forall x$の順であるのに対し,連続性では$\forall x,\, \exists \delta;$の順であるから,一様連続なら連続である。
    \end{proof}
\end{prop}

\begin{cor}\corlabel{Lipschitz-Lipschitz連続なら連続}
    $A$上Lipschitz連続な関数は$A$上連続である。
    \begin{proof}
        \propref{Lipschitz-Lipschitz連続なら一様連続}と
        \propref{Lipschitz-一様連続なら連続}から直ちに従う。
    \end{proof}
\end{cor}

\begin{prop}\proplabel{Lipschitz-原始関数はLipschitz連続}
    $-\infty<a<b<\infty$とし,$I\coloneqq [a,b]$と定め,$f\colon I\to \R^n$は$I$上で有界かつ可積分とする。
    $F\colon I\to \R^n$を,
    \begin{equation}
        \fun{F}{x}\coloneqq \int_a^x dt\, \fun{f}{t}
    \end{equation}
    によって定めると,$F$は$I$上Lipschitz連続である。
    \begin{proof}
        $x,y\in I$に対し,
        \begin{equation}
            \fun{F^j}{x}-\fun{F^j}{y}=\int_a^x dt\, \fun{f^j}{t} - \int_a^y dt\, \fun{f^j}{t}=\int_y^x dt\, \fun{f^j}{t}
        \end{equation}
        である。
        $f$は$I$上で有界であるから,
        \begin{equation}
            L\coloneqq \sup_{t\in I}\bra{\norm{\fun{f}{t}}}\geq 0
        \end{equation}
        と定めておくと,$I$上で常に$\abs{\fun{f^j}{t}}\leq L$であり,\equref{実ノルム-積分絶対値不等式}により,
        \begin{equation}
            \abs{\fun{F^j}{x}-\fun{F^j}{y}}=\abs{\int_y^x dt\, \fun{f^j}{t}}
            \leq \abs{\int_x^y dt\, \abs{\fun{f^j}{t}}}\leq \abs{\int_x^y dt\, L}
            = L\abs{x-y}
        \end{equation}
        である。
        これより,$I$上の任意の$x$, $y$に関して
        \begin{equation}
            \norm{\fun{F}{x}-\fun{F}{y}}\leq L\abs{x-y}
        \end{equation}
        が成立するから,$F$は$I$上Lipschitz連続である。
    \end{proof}
\end{prop}

\begin{prop}\proplabel{Lipschitz-微分が有界ならLipschitz連続}
    $a<b$とする。
    $f\colon [a,b]\to \R^n$は$[a,b]$上で連続,$(a,b)$上で微分可能とし,$(a,b)$上で$\norm{\fun{f'}{t}}$は有限であるとする。
    このとき,$f$は$[a,b]$上でLipschitz連続である。
    \begin{proof}
        $(a,b)$上での$\norm{\fun{f'}{t}}$の有界性から,
        \begin{equation}
            L\coloneqq \sup_{t\in(a,b)}\max_{j}\bra{\abs{\fun{(f^j)'}{t}}}<\infty
        \end{equation}
        が定まり,$L\geq 0$である。
        $f$の各成分について,平均値の定理より,
        \begin{equation}
            \forall j\in \setN, \forall x,\forall y\in[a,b], \squ{
                x<y\, \Rightarrow\, \exists c_j\in (a,b); \pare{\pare{x<c_j<y}\, \land\, \frac{\fun{f^j}{x}-\fun{f^j}{y}}{x-y}=\fun{(f^j)'}{c_j}}
            }
        \end{equation}
        が成立する。
        このような$c_j$を取ったとき,$\fun{(f^j)'}{c_j}\leq L$であるから,
        \begin{equation}
            \norm{\fun{f}{x}-\fun{f}{y}}=\max_{j}\bra{\abs{\fun{f^j}{x}-\fun{f^j}{y}}}
            =\max_{j}\bra{\abs{\fun{(f^j)'}{c_j}}\abs{x-y}}\leq L\abs{x-y}
            \equlabel{Lipschitz-f'が有界ならLipschitz連続-不等式}
        \end{equation}
        である。
        $x>y$のときにも平均値の定理により適当な定数$\tilde{c}_j$の存在が言えるので,\equref{Lipschitz-f'が有界ならLipschitz連続-不等式}は同様に成立する。
        $x=y$のとき,$\norm{0}\leq L\abs{0}$は成立するので\equref{Lipschitz-f'が有界ならLipschitz連続-不等式}も成立する。
        これより,$[a,b]$上の任意の$x$, $y$に関して
        $\norm{\fun{f}{x}-\fun{f}{y}}\leq L\abs{x-y}$の成立が言えた。
        従って,$f$はLipschitz連続である。
    \end{proof}
\end{prop}

\end{document}
